% Template for PLoS
% Version 3.2 March 2016
%
% % % % % % % % % % % % % % % % % % % % % %
%
% -- IMPORTANT NOTE
%
% This template contains comments intended 
% to minimize problems and delays during our production 
% process. Please follow the template instructions
% whenever possible.
%
% % % % % % % % % % % % % % % % % % % % % % % 
%
% Once your paper is accepted for publication, 
% PLEASE REMOVE ALL TRACKED CHANGES in this file 
% and leave only the final text of your manuscript. 
% PLOS recommends the use of latexdiff to track changes during review, as this will help to maintain a clean tex file.
% Visit https://www.ctan.org/pkg/latexdiff?lang=en for info or contact us at latex@plos.org.
%
%
% There are no restrictions on package use within the LaTeX files except that 
% no packages listed in the template may be deleted.
%
% Please do not include colors or graphics in the text.
%
% The manuscript LaTeX source should be contained within a single file (do not use \input, \externaldocument, or similar commands).
%
% % % % % % % % % % % % % % % % % % % % % % %
%
% -- FIGURES AND TABLES
%
% Please include tables/figure captions directly after the paragraph where they are first cited in the text.
%
% DO NOT INCLUDE GRAPHICS IN YOUR MANUSCRIPT
% - Figures should be uploaded separately from your manuscript file. 
% - Figures generated using LaTeX should be extracted and removed from the PDF before submission. 
% - Figures containing multiple panels/subfigures must be combined into one image file before submission.
% For figure citations, please use "Fig" instead of "Figure".
% See http://journals.plos.org/plosone/s/figures for PLOS figure guidelines.
%
% Tables should be cell-based and may not contain:
% - tabs/spacing/line breaks within cells to alter layout or alignment
% - vertically-merged cells (no tabular environments within tabular environments, do not use \multirow)
% - colors, shading, or graphic objects
% See http://journals.plos.org/plosone/s/tables for table guidelines.
%
% For tables that exceed the width of the text column, use the adjustwidth environment as illustrated in the example table in text below.
%
% % % % % % % % % % % % % % % % % % % % % % % %
%
% -- EQUATIONS, MATH SYMBOLS, SUBSCRIPTS, AND SUPERSCRIPTS
%
% IMPORTANT
% Below are a few tips to help format your equations and other special characters according to our specifications. For more tips to help reduce the possibility of formatting errors during conversion, please see our LaTeX guidelines at http://journals.plos.org/plosone/s/latex
%
% For inline equations, please be sure to include all portions of an equation in the math environment.  For example, x$^2$ is incorrect; this should be formatted as $x^2$ (or $\mathrm{x}^2$ if the romanized font is desired).
%
% Do not include text that is not math in the math environment. For example, CO2 should be written as CO\textsubscript{2} instead of CO$_2$.
%
% Please add line breaks to long display equations when possible in order to fit size of the column. 
%
% For inline equations, please do not include punctuation (commas, etc) within the math environment unless this is part of the equation.
%
% When adding superscript or subscripts outside of brackets/braces, please group using {}.  For example, change "[U(D,E,\gamma)]^2" to "{[U(D,E,\gamma)]}^2". 
%
% Do not use \cal for caligraphic font.  Instead, use \mathcal{}
%
% % % % % % % % % % % % % % % % % % % % % % % % 
%
% Please contact latex@plos.org with any questions.
%
% % % % % % % % % % % % % % % % % % % % % % % %

\documentclass[10pt,letterpaper]{article}
\usepackage[top=0.85in,left=2.75in,footskip=0.75in]{geometry}

% Use adjustwidth environment to exceed column width (see example table in text)
\usepackage{changepage}

% Use Unicode characters when possible
\usepackage[utf8x]{inputenc}

% textcomp package and marvosym package for additional characters
\usepackage{textcomp,marvosym}

% fixltx2e package for \textsubscript
\usepackage{fixltx2e}

% amsmath and amssymb packages, useful for mathematical formulas and symbols
\usepackage{amsmath,amssymb}

% cite package, to clean up citations in the main text. Do not remove.
\usepackage{cite}

% Use nameref to cite supporting information files (see Supporting Information section for more info)
\usepackage{nameref,hyperref}

% line numbers
\usepackage[right]{lineno}

% ligatures disabled
\usepackage{microtype}
\DisableLigatures[f]{encoding = *, family = * }

% Remove comment for double spacing
%\usepackage{setspace} 
%\doublespacing

% Text layout
\raggedright
\setlength{\parindent}{0.5cm}
\textwidth 5.25in 
\textheight 8.75in

% Bold the 'Figure #' in the caption and separate it from the title/caption with a period
% Captions will be left justified
\usepackage[aboveskip=1pt,labelfont=bf,labelsep=period,justification=raggedright,singlelinecheck=off]{caption}
\renewcommand{\figurename}{Fig}

% Use the PLoS provided BiBTeX style
\bibliographystyle{plos2015}

% Remove brackets from numbering in List of References
\makeatletter
\renewcommand{\@biblabel}[1]{\quad#1.}
\makeatother

% Leave date blank
\date{}

% Header and Footer with logo
\usepackage{lastpage,fancyhdr,graphicx}
\usepackage{epstopdf}
\pagestyle{myheadings}
\pagestyle{fancy}
\fancyhf{}
\setlength{\headheight}{27.023pt}
\lhead{\includegraphics[width=2.0in]{PLOS-submission.eps}}
\rfoot{\thepage/\pageref{LastPage}}
\renewcommand{\footrule}{\hrule height 2pt \vspace{2mm}}
\fancyheadoffset[L]{2.25in}
\fancyfootoffset[L]{2.25in}
\lfoot{\sf PLOS}

%% Include all macros below

\newcommand{\lorem}{{\bf LOREM}}
\newcommand{\ipsum}{{\bf IPSUM}}

%% END MACROS SECTION


\begin{document}
\vspace*{0.2in}

% Title must be 250 characters or less.
\begin{flushleft}
{\Large
\textbf\newline{Effects of Epidemiological Structure on the Transient Evolution of HIV Virulence} % Please use "title case" (capitalize all terms in the title except conjunctions, prepositions, and articles).
}
\newline
% Insert author names, affiliations and corresponding author email (do not include titles, positions, or degrees).
\\
Name1 Surname\textsuperscript{1,2\Yinyang},
Name2 Surname\textsuperscript{2\Yinyang},
Name3 Surname\textsuperscript{2,3\textcurrency},
Name4 Surname\textsuperscript{2},
Name5 Surname\textsuperscript{2\ddag},
Name6 Surname\textsuperscript{2\ddag},
Name7 Surname\textsuperscript{1,2,3*},
with the Lorem Ipsum Consortium\textsuperscript{\textpilcrow}
\\
\bigskip
\textbf{1} Affiliation Dept/Program/Center, Institution Name, City, State, Country
\\
\textbf{2} Affiliation Dept/Program/Center, Institution Name, City, State, Country
\\
\textbf{3} Affiliation Dept/Program/Center, Institution Name, City, State, Country
\\
\bigskip

% Insert additional author notes using the symbols described below. Insert symbol callouts after author names as necessary.
% 
% Remove or comment out the author notes below if they aren't used.
%
% Primary Equal Contribution Note
\Yinyang These authors contributed equally to this work.

% Additional Equal Contribution Note
% Also use this double-dagger symbol for special authorship notes, such as senior authorship.
\ddag These authors also contributed equally to this work.

% Current address notes
\textcurrency Current Address: Dept/Program/Center, Institution Name, City, State, Country % change symbol to "\textcurrency a" if more than one current address note
% \textcurrency b Insert second current address 
% \textcurrency c Insert third current address

% Deceased author note
\dag Deceased

% Group/Consortium Author Note
\textpilcrow Membership list can be found in the Acknowledgments section.

% Use the asterisk to denote corresponding authorship and provide email address in note below.
* correspondingauthor@institute.edu

\end{flushleft}
% Please keep the abstract below 300 words
\section*{Abstract}
The evolutionary dynamics of parasite virulence over the
  course of an emerging epidemic have important implications both for
  our basic understanding of epidemiological dynamics and,
  potentially, for the outcomes of public health interventions. In
  general changes in fitness landscapes over the course of an epidemic
  will select for higher virulence during the early,
  exponential-growth phase of the epidemic, but quantitative outcomes
  can depend sensitively on biological details and the structure of
  mathematical models used to capture them.  Fraser, Shirreff, and
  co-workers have proposed a series of models for eco-evolutionary
  dynamics of HIV that are relatively detailed in their portrayal of
  the tradeoffs between transmission and virulence (mediated by
  set-point viral load, SPVL) and their heritability between
  hosts. However, these models use very simple implicit
  representations of the transmission process that ignore the
  partnership dynamics that previous research has found to be critical
  in predicting epidemics of sexually transmitted diseases.  We
  explore models that combine HIV virulence tradeoffs with a range of
  epidemiological structures, modeling partnership formation and
  dissolution and allowing for individuals to transmit disease outside
  of partnerships. We assess summary statistics such as the peak value
  of virulence (SPVL) and the time at which the peak occurs across all
  models and across a Latin hypercube sample that captures a realistic
  range of partnership dynamic parameters for sub-Saharan Africa. In
  order to account for the different interpretations of parameters
  across model structures, we scale all parameter sets to constrain
  the simulated epidemic growth rate to be identical, matching a
  realistic baseline value. Our primary result is that, for this
  particular model setting, the simplest random-mixing structure is
  actually the best approximation to the most realistic model; this
  surprising outcome occurs because the dominance of extra-pair
  contact in the realistic model tends to mask the effects of
  partnership structure.

% Please keep the Author Summary between 150 and 200 words
% Use first person. PLOS ONE authors please skip this step. 
% Author Summary not valid for PLOS ONE submissions.   
\section*{Author Summary}
Lorem ipsum dolor sit amet, consectetur adipiscing elit. Curabitur eget porta erat. Morbi consectetur est vel gravida pretium. Suspendisse ut dui eu ante cursus gravida non sed sem. Nullam sapien tellus, commodo id velit id, eleifend volutpat quam. Phasellus mauris velit, dapibus finibus elementum vel, pulvinar non tellus. Nunc pellentesque pretium diam, quis maximus dolor faucibus id. Nunc convallis sodales ante, ut ullamcorper est egestas vitae. Nam sit amet enim ultrices, ultrices elit pulvinar, volutpat risus.

\linenumbers

% Use "Eq" instead of "Equation" for equation citations.
\section*{Introduction}


The evolution of pathogen virulence is a fundamental process in
evolutionary biology, of both theoretical and (potentially) practical
importance. The trade-off theory \cite{Ebert1999} --- which
postulates that parasite virulence can be explained as the long-term
evolutionary outcome of a saturating relationship between parasite
clearance rate and transmission rate --- has been criticized
\cite{EbertBull2003,alizon_adaptive_2015}, but has also been
successfully applied in a variety of host-pathogen systems \cite{Dwyer+1990,mackinnon1999genetic,jensen2006empirical,deroode2008virulence}. One
particularly interesting application of these ideas is the work by
Fraser \textit{et al}. showing that HIV appears to satisfy the prerequisites of
the tradeoff theory: in a study of discordant couples (i.e. long-term
sexual partnerships with one infected and one uninfected partner), HIV
virulence as measured by the rate of progression to AIDS was both
heritable and covaried with the set-point viral load (i.e., the
characteristic virus load measured in blood during the intermediate
stage of infection), which in turn predicted the probability of
within-couple transmission
\cite{Fraser+2007,fraser_virulence_2014}. Subsequent studies
\cite{shirreff_transmission_2011,herbeck_hiv_2014} used these data to
parameterize mechanistic models of HIV virulence evolution, suggesting
that HIV invading a novel population would initially evolve increased
virulence, peaking after approximately [XXX] years and then declining
slightly to a long-stable virulence level.

The work of Shirreff \textit{et al.} \cite{shirreff_transmission_2011}, and particularly the predicted
transient peak in HIV virulence midway through the epidemic,
highlights the importance of interactions between epidemiological and
evolutionary factors \cite{day_virulence_2004,alizon_price_2009}.
However, despite the attention to mechanistic detail at the individual
or physiological level, the epidemiological structures used in these
models are relatively simple.

As we discuss in detail below, the
existing models of HIV eco-evolutionary dynamics either use implicit
models that incorporate the average effects of within-couple sexual
contact --- without representing the explicit dynamics of pair
formation and dissolution or accounting for extra-partnership contact
--- or use an agent-based formulation with parameters that effectively
lead to random mixing among infected and uninfected individuals. Here
we explore the effects of incorporating \emph{explicit}
epidemiological structure in eco-evolutionary models.

We add complexity to the epidemiological model following the general approach of Champredon \textit{et al.} \cite{champredon_hiv_2013}; individuals join and leave partnerships at a specified rate, and can have sexual contact both within and outside of established partnerships. At the same time, our analysis somewhat simplifies the models of Shirreff \textit{et al.} \cite{shirreff_transmission_2011}, for computational tractability; we check that our qualitative results are not sensitive to these simplifications. In order to explore how virulence evolution depends on epidemiological structure, we consider a series of models with increasing levels of complexity. In order to avoid dependence of the results on a particular set of parameters --- as we explain below, finding matching sets of parameters across models with widely differing epidemiological structures is challenging --- we evaluate our models across a wide range of parameters, again following Champredon \textit{et al.} \cite{champredon_hiv_2013} in using a Latin hypercube design. For each model run, we compute a set of metrics (peak virulence, timing of virulence peak, equilibrium virulence) that summarize the evolutionary trajectory of a simulated HIV epidemic.

\section*{Materials and Methods}

As our primary goal is to explore how different epidemiological structures (i.e. partnership dynamics and contact structures) affect our conclusions about the evolution of virulence, our models use a simplified description of within-host dynamics and heritability derived from 
Shirreff \textit{et al.}'s multi-strain evolutionary model \cite{shirreff_transmission_2011}. Like Shirreff \textit{et al.}, we use a simple susceptible-infected-susceptible demographic formulation; rather than modeling birth and death (or more specifically, recruitment into the sexually active population and death), we assume that whenever an individual dies from infection, another enters the susceptible compartment.

\subsection*{Infection dynamics}

Like Shirreff \textit{et al.} \cite{shirreff_transmission_2011}, we focus on the evolution of mean $\log_{10}$ set-point viral load, SPVL (which we denote as $\alpha$), rather than virulence (i.e. rate of progression to AIDS) itself. 
In contrast to Shirreff \textit{et al.}, we use a single-stage disease model instead of accounting explicitly for progression through the three main stages of HIV infection (primary, asymptomatic, and disease), and we use a simple exponentially distributed infectious period instead of a more realistic Weibull-distributed infectious period. We account for varying transmission rates and durations of each disease stage by summing the durations of three stages (again based on Shirreff \textit{et al.}'s model) and taking the duration-weighted average of transmission rates of three stages. Thus the within-couple transmission rate, $\beta$, for our models is given by:
\begin{equation}
\label{eq:1}
\beta (\alpha) = \frac{D_P \beta_P + D_A (\alpha) \beta_A (\alpha) + D_D \beta_D}{D_P + D_A (\alpha) + D_D}
\end{equation}
where the duration of infection ($D_P$ and $D_D$) and rate of transmission ($\beta_P$ and $\beta_D$) of the \textbf{P}rimary and \textbf{D}isease stages
of infection are independent of the host's SPVL. Following Shirreff \textit{et al.}, the duration of infection ($D_A$) and rate of transmission ($\beta_A$) for the \textbf{A}symptomatic stage are Hill functions of the SPVL:

\begin{equation}
\label{eq:2}
\begin{split}
D_A(\alpha) &= \frac{D_{max} D_{50} ^{D_k}}{V_\alpha ^{D_k} + D_{50}^{D_k}}, \\
\beta_A(\alpha) &= \frac{\beta_{max} V_\alpha ^ {\beta_k}}%
{V_\alpha^{\beta_k} + \beta_{50} ^{\beta_k}},
\end{split}
\end{equation}
where $V_{\alpha} = 10^\alpha$. 
The \textbf{u}ncoupled and \textbf{e}xtra-couple transmission rates are scaled by multiplying the \textbf{w}ithin-couple transmission rate $\beta$ by the contact ratios $c_u/c_w$ and $c_e/c_w$.

\subsection*{Mutation}

Like Shirreff \textit{et al.} \cite{shirreff_transmission_2011} we incorporate a between-host mutation process in the SPVL, but simplify Shirreff \textit{et al.}'s evolutionary model slightly by using a one-to-one genotype-phenotype mapping. 
The mutational process in our model is directly taken from Shirreff \textit{et al.}. Over the course of infection, mutation occurs within the host. However, it is assumed that SPVL of an infected individual is determined by the SPVL at the time of infection for simplicity (and is not further affected by within-host mutation). Instead, the mutational effect takes place when an infected individual transmits the virus to a susceptible individual. First, the distribution of $\log_{10}$ SPVL is discretized into a vector:
\begin{equation}
\label{eq:3}
\alpha_i =  (\alpha_{max} - \alpha_{min})\frac{(i-1)}{n-1} + \alpha_{min} \qquad i = 1,2,3, \dots n.
\end{equation}
We have experimented with varying degrees of discretization in the strain distribution (i.e., values of $n$); in our model runs comparing results with Shirreff \textit{et al.} \cite{shirreff_transmission_2011} (Figure~\ref{fig:panel3}) we use $n=51$ (i.e. a bin width of 0.05 $\log_{10}$ SPVL for $\alpha$), but we find only small differences when reducing $n$ to 21 (bin width 0.25 $\log_{10}$ SPVL), which we use for all other simulations.

We construct an $n$ by $n$ mutational matrix, $M$ --- which is multiplied with the transmission term ---  so that $M_{ij}$ is the probability that a newly infected individual will have $\log_{10}$ SPVL of $\alpha_j$ given that the infector has $\log_{10}$ SPVL of $\alpha_i$. Finally, the probabilities are normalized so that each row sums to 1:
\begin{equation}
M_{ij} = \frac{\Phi(\alpha_j + d/2;i) - \Phi(\alpha_j - d/2;i)}{\Phi(\alpha_{max} + d/2;i) - \Phi(\alpha_{min} - d/2;i)},
\end{equation}
where $\Phi(x;i)$ is the Gaussian cumulative distribution function with mean $\alpha_i$ and variance of $\sigma_M^2$, and $d = (\alpha_{max} - \alpha_{min})/(n-1)$. Transmission rate and disease induced mortality rates are discretized into a vector as well:
\begin{equation}
\begin{aligned}
\beta_i &= \beta(\alpha_i)\\
\lambda_i &= \frac{1}{D_P + D_A (\alpha_i) + D_D}
\end{aligned}
\end{equation}

\subsection*{Contact structure and partnership dynamics}

We developed six multi-strain evolutionary models, designed to cover a gamut between Champredon text{et al.}'s relatively realistic \cite{champredon_hiv_2013} and Shirreff \textit{et al.}'s relatively simplistic \cite{shirreff_transmission_2011} epidemiological structures, each of which is based on different assumptions regarding contact structure and partnership dynamics. Specifically, we focus on the effects of the assumptions of (1) instantaneous vs. non-instantaneous partnership formation and (2) zero vs. positive extra-partnership sexual contact and transmission on the evolution of mean $\log_{10}$ SPVL.

Our first four models consider explicit partnership dynamics and are based on Champredon text{et al.}'s model \cite{champredon_hiv_2013}. Models 1 and 2 assume non-instantaneous partnership formation (i.e. individuals spend some time uncoupled, outside of partnerships) and consist of five states that are classified by infection status and partnership status. $S$ is the number of single (uncoupled) susceptible individuals, and $I$ is the number of single infected individuals. $SS$ is the number of susceptible-susceptible couples, $SI$ is the number of serodiscordant (susceptible-infected) couples, and $II$ is the number of concordant positive (infected-infected) couples. Model 1 includes extra-partnership contact (with both uncoupled individuals and individuals in other partnerships) whereas model 2 only considers within-couple transmission. Models 3 and 4 assume instantaneous partnership formation and thus consist of only the three partnered states: $SS$, $SI$, and $II$. Parallel to model 1 and 2, model 3 includes extra-partnership contact (now only with individuals in other partnerships, since uncoupled individuals don't exist in this model) and model 4 only considers within-couple transmission.

In contrast, models 5 and 6 are not explicitly structured.  Model 5 is an implicit serial monogamy model based on the epidemiological model used by Shirreff \textit{et al.} \cite{shirreff_transmission_2011}. It is actually a random mixing model that consist of only two states, $S$ and $I$, and does not consider explicit partnership dynamics. However, to simulate the effect of instantaneous partnership formation, it uses an adjusted transmission rate that is derived from approximated basic reproduction number of a serial monogamy model \cite{hollingsworth_hiv1_2008}. Finally, model 6 is a simple random-mixing model.

The base model (i.e. model 1) for the first four models is an extension of Champredon text{et al.}'s model \cite{champredon_hiv_2013}. Individuals in single compartment acquire a partner at a rate, $\rho$, and partnerships dissolve at a rate, $c$. Infected individuals in a discordant partnership infect susceptible partner at a rate $\beta$ (within-couple transmission rate) and susceptible individuals outside the partnership at a rate $c_e$ (extra-couple transmission rate). Likewise, a single infected individual can infect any susceptible individuals at a rate $c_u$ through uncoupled mixing. Extra-couple and uncoupled transmission are modeled in a same way as Champredon text{et al.}'s model. All the details have been adapted to a multi-strain scenario. Model 2, 3, and 4 are derived from the base model by removing epidemiological details (partnership formation and uncoupled/extra-couple contact). Model details are explained in the appendix.

% Results and Discussion can be combined.
\section*{Results}
Nulla mi mi, venenatis sed ipsum varius, Table~\ref{table1} volutpat euismod diam. Proin rutrum vel massa non gravida. Quisque tempor sem et dignissim rutrum. Lorem ipsum dolor sit amet, consectetur adipiscing elit. Morbi at justo vitae nulla elementum commodo eu id massa. In vitae diam ac augue semper tincidunt eu ut eros. Fusce fringilla erat porttitor lectus cursus, vel sagittis arcu lobortis. Aliquam in enim semper, aliquam massa id, cursus neque. Praesent faucibus semper libero.

% Place tables after the first paragraph in which they are cited.
\begin{table}[!ht]
\begin{adjustwidth}{-2.25in}{0in} % Comment out/remove adjustwidth environment if table fits in text column.
\centering
\caption{
{\bf Table caption Nulla mi mi, venenatis sed ipsum varius, volutpat euismod diam.}}
\begin{tabular}{|l|l|l|l|l|l|l|l|}
\hline
\multicolumn{4}{|l|}{\bf Heading1} & \multicolumn{4}{|l|}{\bf Heading2}\\ \hline
$cell1 row1$ & cell2 row 1 & cell3 row 1 & cell4 row 1 & cell5 row 1 & cell6 row 1 & cell7 row 1 & cell8 row 1\\ \hline
$cell1 row2$ & cell2 row 2 & cell3 row 2 & cell4 row 2 & cell5 row 2 & cell6 row 2 & cell7 row 2 & cell8 row 2\\ \hline
$cell1 row3$ & cell2 row 3 & cell3 row 3 & cell4 row 3 & cell5 row 3 & cell6 row 3 & cell7 row 3 & cell8 row 3\\ \hline
\end{tabular}
\begin{flushleft} Table notes Phasellus venenatis, tortor nec vestibulum mattis, massa tortor interdum felis, nec pellentesque metus tortor nec nisl. Ut ornare mauris tellus, vel dapibus arcu suscipit sed.
\end{flushleft}
\label{table1}
\end{adjustwidth}
\end{table}


%PLOS does not support heading levels beyond the 3rd (no 4th level headings).
\subsection*{\lorem\ and \ipsum\ Nunc blandit a tortor.}
\subsubsection*{3rd Level Heading.} 
Maecenas convallis mauris sit amet sem ultrices gravida. Etiam eget sapien nibh. Sed ac ipsum eget enim egestas ullamcorper nec euismod ligula. Curabitur fringilla pulvinar lectus consectetur pellentesque. Quisque augue sem, tincidunt sit amet feugiat eget, ullamcorper sed velit. Sed non aliquet felis. Lorem ipsum dolor sit amet, consectetur adipiscing elit. Mauris commodo justo ac dui pretium imperdiet. Sed suscipit iaculis mi at feugiat. 

\begin{enumerate}
	\item{react}
	\item{diffuse free particles}
	\item{increment time by dt and go to 1}
\end{enumerate}


\subsection*{Sed ac quam id nisi malesuada congue.}

Nulla mi mi, venenatis sed ipsum varius, volutpat euismod diam. Proin rutrum vel massa non gravida. Quisque tempor sem et dignissim rutrum. Lorem ipsum dolor sit amet, consectetur adipiscing elit. Morbi at justo vitae nulla elementum commodo eu id massa. In vitae diam ac augue semper tincidunt eu ut eros. Fusce fringilla erat porttitor lectus cursus, vel sagittis arcu lobortis. Aliquam in enim semper, aliquam massa id, cursus neque. Praesent faucibus semper libero.

\begin{itemize}
	\item First bulleted item.
	\item Second bulleted item.
	\item Third bulleted item.
\end{itemize}

\section*{Discussion}
Nulla mi mi, venenatis sed ipsum varius, Table~\ref{table1} volutpat euismod diam. Proin rutrum vel massa non gravida. Quisque tempor sem et dignissim rutrum. Lorem ipsum dolor sit amet, consectetur adipiscing elit. Morbi at justo vitae nulla elementum commodo eu id massa. In vitae diam ac augue semper tincidunt eu ut eros. Fusce fringilla erat porttitor lectus cursus, vel sagittis arcu lobortis. Aliquam in enim semper, aliquam massa id, cursus neque. Praesent faucibus semper libero~\cite{bib3}.

\section*{Conclusion}

CO\textsubscript{2} Maecenas convallis mauris sit amet sem ultrices gravida. Etiam eget sapien nibh. Sed ac ipsum eget enim egestas ullamcorper nec euismod ligula. Curabitur fringilla pulvinar lectus consectetur pellentesque. Quisque augue sem, tincidunt sit amet feugiat eget, ullamcorper sed velit. 

Sed non aliquet felis. Lorem ipsum dolor sit amet, consectetur adipiscing elit. Mauris commodo justo ac dui pretium imperdiet. Sed suscipit iaculis mi at feugiat. Ut neque ipsum, luctus id lacus ut, laoreet scelerisque urna. Phasellus venenatis, tortor nec vestibulum mattis, massa tortor interdum felis, nec pellentesque metus tortor nec nisl. Ut ornare mauris tellus, vel dapibus arcu suscipit sed. Nam condimentum sem eget mollis euismod. Nullam dui urna, gravida venenatis dui et, tincidunt sodales ex. Nunc est dui, sodales sed mauris nec, auctor sagittis leo. Aliquam tincidunt, ex in facilisis elementum, libero lectus luctus est, non vulputate nisl augue at dolor. For more information, see \nameref{S1_Appendix}.

\section*{Supporting Information}

% Include only the SI item label in the paragraph heading. Use the \nameref{label} command to cite SI items in the text.
\paragraph*{S1 Fig.}
\label{S1_Fig}
{\bf Bold the title sentence.} Add descriptive text after the title of the item (optional).

\paragraph*{S2 Fig.}
\label{S2_Fig}
{\bf Lorem Ipsum.} Maecenas convallis mauris sit amet sem ultrices gravida. Etiam eget sapien nibh. Sed ac ipsum eget enim egestas ullamcorper nec euismod ligula. Curabitur fringilla pulvinar lectus consectetur pellentesque.

\paragraph*{S1 File.}
\label{S1_File}
{\bf Lorem Ipsum.}  Maecenas convallis mauris sit amet sem ultrices gravida. Etiam eget sapien nibh. Sed ac ipsum eget enim egestas ullamcorper nec euismod ligula. Curabitur fringilla pulvinar lectus consectetur pellentesque.

\paragraph*{S1 Video.}
\label{S1_Video}
{\bf Lorem Ipsum.}  Maecenas convallis mauris sit amet sem ultrices gravida. Etiam eget sapien nibh. Sed ac ipsum eget enim egestas ullamcorper nec euismod ligula. Curabitur fringilla pulvinar lectus consectetur pellentesque.

\paragraph*{S1 Appendix.}
\label{S1_Appendix}
{\bf Lorem Ipsum.} Maecenas convallis mauris sit amet sem ultrices gravida. Etiam eget sapien nibh. Sed ac ipsum eget enim egestas ullamcorper nec euismod ligula. Curabitur fringilla pulvinar lectus consectetur pellentesque.

\paragraph*{S1 Table.}
\label{S1_Table}
{\bf Lorem Ipsum.} Maecenas convallis mauris sit amet sem ultrices gravida. Etiam eget sapien nibh. Sed ac ipsum eget enim egestas ullamcorper nec euismod ligula. Curabitur fringilla pulvinar lectus consectetur pellentesque.

\section*{Acknowledgments}
Cras egestas velit mauris, eu mollis turpis pellentesque sit amet. Interdum et malesuada fames ac ante ipsum primis in faucibus. Nam id pretium nisi. Sed ac quam id nisi malesuada congue. Sed interdum aliquet augue, at pellentesque quam rhoncus vitae.

\nolinenumbers

% Either type in your references using
% \begin{thebibliography}{}
% \bibitem{}
% Text
% \end{thebibliography}
%
% or
%
% Compile your BiBTeX database using our plos2015.bst
% style file and paste the contents of your .bbl file
% here.
% 


\bibliographystyle{plos2015}
\bibliography{virulence}

\end{document}

