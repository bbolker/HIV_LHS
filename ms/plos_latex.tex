% Template for PLoS
% Version 3.2 March 2016
%
% % % % % % % % % % % % % % % % % % % % % %
%
% -- IMPORTANT NOTE
%
% This template contains comments intended 
% to minimize problems and delays during our production 
% process. Please follow the template instructions
% whenever possible.
%
% % % % % % % % % % % % % % % % % % % % % % % 
%
% Once your paper is accepted for publication, 
% PLEASE REMOVE ALL TRACKED CHANGES in this file 
% and leave only the final text of your manuscript. 
% PLOS recommends the use of latexdiff to track changes during review, as this will help to maintain a clean tex file.
% Visit https://www.ctan.org/pkg/latexdiff?lang=en for info or contact us at latex@plos.org.
%
%
% There are no restrictions on package use within the LaTeX files except that 
% no packages listed in the template may be deleted.
%
% Please do not include colors or graphics in the text.
%
% The manuscript LaTeX source should be contained within a single file (do not use \input, \externaldocument, or similar commands).
%
% % % % % % % % % % % % % % % % % % % % % % %
%
% -- FIGURES AND TABLES
%
% Please include tables/figure captions directly after the paragraph where they are first cited in the text.
%
% DO NOT INCLUDE GRAPHICS IN YOUR MANUSCRIPT
% - Figures should be uploaded separately from your manuscript file. 
% - Figures generated using LaTeX should be extracted and removed from the PDF before submission. 
% - Figures containing multiple panels/subfigures must be combined into one image file before submission.
% For figure citations, please use "Fig" instead of "Figure".
% See http://journals.plos.org/plosone/s/figures for PLOS figure guidelines.
%
% Tables should be cell-based and may not contain:
% - tabs/spacing/line breaks within cells to alter layout or alignment
% - vertically-merged cells (no tabular environments within tabular environments, do not use \multirow)
% - colors, shading, or graphic objects
% See http://journals.plos.org/plosone/s/tables for table guidelines.
%
% For tables that exceed the width of the text column, use the adjustwidth environment as illustrated in the example table in text below.
%
% % % % % % % % % % % % % % % % % % % % % % % %
%
% -- EQUATIONS, MATH SYMBOLS, SUBSCRIPTS, AND SUPERSCRIPTS
%
% IMPORTANT
% Below are a few tips to help format your equations and other special characters according to our specifications. For more tips to help reduce the possibility of formatting errors during conversion, please see our LaTeX guidelines at http://journals.plos.org/plosone/s/latex
%
% For inline equations, please be sure to include all portions of an equation in the math environment.  For example, x$^2$ is incorrect; this should be formatted as $x^2$ (or $\mathrm{x}^2$ if the romanized font is desired).
%
% Do not include text that is not math in the math environment. For example, CO2 should be written as CO\textsubscript{2} instead of CO$_2$.
%
% Please add line breaks to long display equations when possible in order to fit size of the column. 
%
% For inline equations, please do not include punctuation (commas, etc) within the math environment unless this is part of the equation.
%
% When adding superscript or subscripts outside of brackets/braces, please group using {}.  For example, change "[U(D,E,\gamma)]^2" to "{[U(D,E,\gamma)]}^2". 
%
% Do not use \cal for caligraphic font.  Instead, use \mathcal{}
%
% % % % % % % % % % % % % % % % % % % % % % % % 
%
% Please contact latex@plos.org with any questions.
%
% % % % % % % % % % % % % % % % % % % % % % % %

\documentclass[10pt,letterpaper]{article}
\usepackage[top=0.85in,left=2.75in,footskip=0.75in]{geometry}

% Use adjustwidth environment to exceed column width (see example table in text)
\usepackage{changepage}

% Use Unicode characters when possible
\usepackage[utf8x]{inputenc}

% textcomp package and marvosym package for additional characters
\usepackage{textcomp,marvosym}

% fixltx2e package for \textsubscript
\usepackage{fixltx2e}

% amsmath and amssymb packages, useful for mathematical formulas and symbols
\usepackage{amsmath,amssymb}

% cite package, to clean up citations in the main text. Do not remove.
\usepackage{cite}

% Use nameref to cite supporting information files (see Supporting Information section for more info)
\usepackage{nameref,hyperref}

% line numbers
\usepackage[right]{lineno}

% ligatures disabled
\usepackage{microtype}
\DisableLigatures[f]{encoding = *, family = * }

% Remove comment for double spacing
%\usepackage{setspace} 
%\doublespacing

% Text layout
\raggedright
\setlength{\parindent}{0.5cm}
\textwidth 5.25in 
\textheight 8.75in

% Bold the 'Figure #' in the caption and separate it from the title/caption with a period
% Captions will be left justified
\usepackage[aboveskip=1pt,labelfont=bf,labelsep=period,justification=raggedright,singlelinecheck=off]{caption}
\renewcommand{\figurename}{Fig}

% Use the PLoS provided BiBTeX style
\bibliographystyle{plos2015}

% Remove brackets from numbering in List of References
\makeatletter
\renewcommand{\@biblabel}[1]{\quad#1.}
\makeatother

% Leave date blank
\date{}

% Header and Footer with logo
\usepackage{lastpage,fancyhdr,graphicx}
\usepackage{epstopdf}
\pagestyle{myheadings}
\pagestyle{fancy}
\fancyhf{}
\setlength{\headheight}{27.023pt}
\lhead{\includegraphics[width=2.0in]{PLOS-submission.eps}}
\rfoot{\thepage/\pageref{LastPage}}
\renewcommand{\footrule}{\hrule height 2pt \vspace{2mm}}
\fancyheadoffset[L]{2.25in}
\fancyfootoffset[L]{2.25in}
\lfoot{\sf PLOS}

%% Include all macros below
\newcommand{\tsub}[2]{#1_{{\textrm{\tiny #2}}}}
\newcommand{\khalf}{\left(\frac{1}{2}\right)^{\delta_{ij}}}  % (1/2)^kronecker
\newcommand{\Lspvl}{$\log_{10}$ SPVL}
\newcommand{\rzero}{{\mathcal R}_0}
\newcommand{\etal}{\textit{et al.}}


%% END MACROS SECTION


\begin{document}
\vspace*{0.2in}

% Title must be 250 characters or less.
\begin{flushleft}
{\Large
\textbf\newline{Effects of Epidemiological Structure on the Transient Evolution of HIV Virulence} % Please use "title case" (capitalize all terms in the title except conjunctions, prepositions, and articles).
}
\newline
% Insert author names, affiliations and corresponding author email (do not include titles, positions, or degrees).
\\
Sang Woo Park\textsuperscript{1}
Benjamin M. Bolker\textsuperscript{2,3,4,*}
\\
\bigskip
\textbf{1} Program in Health Sciences, McMaster University, Hamilton, Ontario, Canada
\\
\textbf{2} Department of Mathematics \& Statistics,  McMaster University, Hamilton, Ontario, Canada
\\
\textbf{3} Department of Biology,  McMaster University, Hamilton, Ontario, Canada
\\
\textbf{4} Institute for Infectious Disease Research,  McMaster University, Hamilton, Ontario, Canada
\\
\bigskip

% Insert additional author notes using the symbols described below. Insert symbol callouts after author names as necessary.
% 
% Remove or comment out the author notes below if they aren't used.
%
% Primary Equal Contribution Note
%\Yinyang These authors contributed equally to this work.

% Additional Equal Contribution Note
% Also use this double-dagger symbol for special authorship notes, such as senior authorship.
%\ddag These authors also contributed equally to this work.

% Current address notes
%\textcurrency Current Address: Dept/Program/Center, Institution Name, City, State, Country % change symbol to "\textcurrency a" if more than one current address note
% \textcurrency b Insert second current address 
% \textcurrency c Insert third current address

% Deceased author note
%\dag Deceased

% Group/Consortium Author Note
%\textpilcrow Membership list can be found in the Acknowledgments section.

% Use the asterisk to denote corresponding authorship and provide email address in note below.
* bolker@mcmaster.ca

\end{flushleft}
% Please keep the abstract below 300 words
% current is 298!
\section*{Abstract}
The evolutionary dynamics of parasite virulence over the
  course of an emerging epidemic have important implications both for
  our basic understanding of epidemiological dynamics and,
  potentially, for the outcomes of public health interventions. In
  general changes in fitness landscapes over the course of an epidemic
  will select for higher virulence during the early,
  exponential-growth phase of the epidemic, but quantitative outcomes
  can depend sensitively on biological details and the structure of
  mathematical models used to capture them.  Fraser, Shirreff, and
  co-workers have proposed a series of models for eco-evolutionary
  dynamics of HIV that are relatively detailed in their portrayal of
  the tradeoffs between transmission and virulence (mediated by
  set-point viral load, SPVL) and their heritability between
  hosts. However, these models use very simple implicit
  representations of the transmission process that ignore the
  partnership dynamics that previous research has found to be critical
  in predicting epidemics of sexually transmitted diseases.  We
  explore models that combine HIV virulence tradeoffs with a range of
  epidemiological structures, modeling partnership formation and
  dissolution and allowing for individuals to transmit disease outside
  of partnerships. We assess summary statistics such as the peak value
  of virulence (SPVL) and the time at which the peak occurs across all
  models and across a Latin hypercube sample that captures a realistic
  range of partnership dynamic parameters for sub-Saharan Africa. In
  order to account for the different interpretations of parameters
  across model structures, we scale all parameter sets to constrain
  the simulated epidemic growth rate to be identical, matching a
  realistic baseline value. Our primary result is that, for this
  particular model setting, the simplest random-mixing structure is
  actually the best approximation to the most realistic model; this
  surprising outcome occurs because the dominance of extra-pair
  contact in the realistic model tends to mask the effects of
  partnership structure.

% Please keep the Author Summary between 150 and 200 words
% Use first person. PLOS ONE authors please skip this step. 
% Author Summary not valid for PLOS ONE submissions.   
\section*{Author Summary}
Lorem ipsum dolor sit amet, consectetur adipiscing elit. Curabitur eget porta erat. Morbi consectetur est vel gravida pretium. Suspendisse ut dui eu ante cursus gravida non sed sem. Nullam sapien tellus, commodo id velit id, eleifend volutpat quam. Phasellus mauris velit, dapibus finibus elementum vel, pulvinar non tellus. Nunc pellentesque pretium diam, quis maximus dolor faucibus id. Nunc convallis sodales ante, ut ullamcorper est egestas vitae. Nam sit amet enim ultrices, ultrices elit pulvinar, volutpat risus.

\linenumbers

% Use "Eq" instead of "Equation" for equation citations.
\section*{Introduction}


The evolution of pathogen virulence is a fundamental process in
evolutionary biology, of both theoretical and (potentially) practical
importance. The trade-off theory \cite{Ebert1999} --- which
postulates that parasite virulence can be explained as the long-term
evolutionary outcome of a saturating relationship between parasite
clearance rate and transmission rate --- has been criticized
\cite{EbertBull2003,alizon_adaptive_2015}, but has also been
successfully applied in a variety of host-pathogen systems \cite{Dwyer+1990,mackinnon1999genetic,jensen2006empirical,deroode2008virulence}. One
particularly interesting application of these ideas is the work by
Fraser \etal\ showing that HIV appears to satisfy the prerequisites of
the tradeoff theory: in a study of discordant couples (i.e. long-term
sexual partnerships with one infected and one uninfected partner), HIV
virulence as measured by the rate of progression to AIDS was both
heritable and covaried with the set-point viral load (i.e., the
characteristic virus load measured in blood during the intermediate
stage of infection), which in turn predicted the probability of
within-couple transmission
\cite{Fraser+2007,fraser_virulence_2014}. Subsequent studies
\cite{shirreff_transmission_2011,herbeck_hiv_2014} used these data to
parameterize mechanistic models of HIV virulence evolution, suggesting
that HIV invading a novel population would initially evolve increased
virulence, peaking after approximately [XXX] years and then declining
slightly to a long-stable virulence level.

The work of Shirreff \etal\ \cite{shirreff_transmission_2011}, and particularly the predicted
transient peak in HIV virulence midway through the epidemic,
highlights the importance of interactions between epidemiological and
evolutionary factors \cite{day_virulence_2004,alizon_price_2009}.
However, despite the attention to mechanistic detail at the individual
or physiological level, the epidemiological structures used in these
models are relatively simple.

As we discuss in detail below, the
existing models of HIV eco-evolutionary dynamics either use implicit
models that incorporate the average effects of within-couple sexual
contact --- without representing the explicit dynamics of pair
formation and dissolution or accounting for extra-partnership contact
--- or use an agent-based formulation with parameters that effectively
lead to random mixing among infected and uninfected individuals. Here
we explore the effects of incorporating \emph{explicit}
epidemiological structure in eco-evolutionary models.

We add complexity to the epidemiological model following the general approach of Champredon \etal\ \cite{champredon_hiv_2013}; individuals join and leave partnerships at a specified rate, and can have sexual contact both within and outside of established partnerships. At the same time, our analysis somewhat simplifies the models of Shirreff \etal\ \cite{shirreff_transmission_2011}, for computational tractability; we check that our qualitative results are not sensitive to these simplifications. In order to explore how virulence evolution depends on epidemiological structure, we consider a series of models with increasing levels of complexity. In order to avoid dependence of the results on a particular set of parameters --- as we explain below, finding matching sets of parameters across models with widely differing epidemiological structures is challenging --- we evaluate our models across a wide range of parameters, again following Champredon \etal\ \cite{champredon_hiv_2013} in using a Latin hypercube design. For each model run, we compute a set of metrics (peak virulence, timing of virulence peak, equilibrium virulence) that summarize the evolutionary trajectory of a simulated HIV epidemic.

As our primary goal is to explore how different epidemiological structures (i.e. partnership dynamics and contact structures) affect our conclusions about the evolution of virulence, our models use a simplified description of within-host dynamics and heritability derived from 
Shirreff \etal's multi-strain evolutionary model \cite{shirreff_transmission_2011}. Like Shirreff \etal, we use a simple susceptible-infected-susceptible demographic formulation; rather than modeling birth and death (or more specifically, recruitment into the sexually active population and death), we assume that whenever an individual dies from infection, another enters the susceptible compartment.

\subsection*{Infection dynamics}

Like Shirreff \etal\ \cite{shirreff_transmission_2011}, we focus on the evolution of mean $\log_{10}$ set-point viral load, SPVL (which we denote as $\alpha$), rather than virulence (i.e. rate of progression to AIDS) itself. 
In contrast to Shirreff \etal, we use a single-stage disease model instead of accounting explicitly for progression through the three main stages of HIV infection (primary, asymptomatic, and disease), and we use a simple exponentially distributed infectious period instead of a more realistic Weibull-distributed infectious period. We account for varying transmission rates and durations of each disease stage by summing the durations of three stages (again based on Shirreff \etal's model) and taking the duration-weighted average of transmission rates of three stages. Thus the within-couple transmission rate, $\beta$, for our models is given by:
\begin{equation}
\beta (\alpha) = \frac{D_P \beta_P + D_A (\alpha) \beta_A (\alpha) + D_D \beta_D}{D_P + D_A (\alpha) + D_D}
\end{equation}
where the duration of infection ($D_P$ and $D_D$) and rate of transmission ($\beta_P$ and $\beta_D$) of the \textbf{P}rimary and \textbf{D}isease stages
of infection are independent of the host's SPVL. Following Shirreff \textit{et al.}, the duration of infection ($D_A$) and rate of transmission ($\beta_A$) for the \textbf{A}symptomatic stage are Hill functions of the SPVL:

\begin{equation}
\begin{split}
D_A(\alpha) &= \frac{D_{max} D_{50} ^{D_k}}{V_\alpha ^{D_k} + D_{50}^{D_k}}, \\
\beta_A(\alpha) &= \frac{\beta_{max} V_\alpha ^ {\beta_k}}%
{V_\alpha^{\beta_k} + \beta_{50} ^{\beta_k}},
\end{split}
\end{equation}
where $V_{\alpha} = 10^\alpha$. 
The \textbf{u}ncoupled and \textbf{e}xtra-couple transmission rates are scaled by multiplying the \textbf{w}ithin-couple transmission rate $\beta$ by the contact ratios $c_u/c_w$ and $c_e/c_w$.

\subsection*{Mutation}

Like Shirreff \etal\ \cite{shirreff_transmission_2011} we incorporate a between-host mutation process in the SPVL, but simplify Shirreff \etal's evolutionary model slightly by using a one-to-one genotype-phenotype mapping. 
The mutational process in our model is directly taken from Shirreff \etal. Over the course of infection, mutation occurs within the host. However, it is assumed that SPVL of an infected individual is determined by the SPVL at the time of infection for simplicity (and is not further affected by within-host mutation). Instead, the mutational effect takes place when an infected individual transmits the virus to a susceptible individual. First, the distribution of \Lspvl\ is discretized into a vector:
\begin{equation}
\alpha_i =  (\alpha_{max} - \alpha_{min})\frac{(i-1)}{n-1} + \alpha_{min} \qquad i = 1,2,3, \dots n.
\end{equation}
We have experimented with varying degrees of discretization in the strain distribution (i.e., values of $n$); in our model runs comparing results with Shirreff \etal\ \cite{shirreff_transmission_2011} (Fig.~\ref{fig:panel3}) we use $n=51$ (i.e. a bin width of 0.05 \Lspvl\ for $\alpha$), but we find only small differences when reducing $n$ to 21 (bin width 0.25 \Lspvl), which we use for all other simulations.

\begin{figure}[!h]
\caption{{\bf Baseline dynamics.}
Time series of mean population virulence (\Lspvl). (a) Shirreff model, effects of varying $r$. (b) Effects of varying initial infectious density $I(0)$. (c) Effects of varying initial mean virulence $\alpha(0)$.}
\label{fig:panel3}
\end{figure}

We construct an $n$ by $n$ mutational matrix, $M$ --- which is multiplied with the transmission term ---  so that $M_{ij}$ is the probability that a newly infected individual will have \Lspvl\ of $\alpha_j$ given that the infector has \Lspvl\ of $\alpha_i$. Finally, the probabilities are normalized so that each row sums to 1:
\begin{equation}
M_{ij} = \frac{\Phi(\alpha_j + d/2;i) - \Phi(\alpha_j - d/2;i)}{\Phi(\alpha_{max} + d/2;i) - \Phi(\alpha_{min} - d/2;i)},
\end{equation}
where $\Phi(x;i)$ is the Gaussian cumulative distribution function with mean $\alpha_i$ and variance of $\sigma_M^2$, and $d = (\alpha_{max} - \alpha_{min})/(n-1)$. Transmission rate and disease induced mortality rates are discretized into a vector as well:
\begin{equation}
\begin{aligned}
\beta_i &= \beta(\alpha_i)\\
\lambda_i &= \frac{1}{D_P + D_A (\alpha_i) + D_D}
\end{aligned}
\end{equation}

\subsection*{Contact structure and partnership dynamics}

We developed six multi-strain evolutionary models, designed to cover a gamut between Champredon \etal's relatively realistic \cite{champredon_hiv_2013} and Shirreff \etal's relatively simplistic \cite{shirreff_transmission_2011} epidemiological structures, each of which is based on different assumptions regarding contact structure and partnership dynamics. Specifically, we focus on the effects of the assumptions of (1) instantaneous vs. non-instantaneous partnership formation and (2) zero vs. positive extra-partnership sexual contact and transmission on the evolution of mean \Lspvl.

Our first four models consider explicit partnership dynamics and are based on Champredon text{et al.}'s model \cite{champredon_hiv_2013}. Models 1 and 2 assume non-instantaneous partnership formation (i.e. individuals spend some time uncoupled, outside of partnerships) and consist of five states that are classified by infection status and partnership status. $S$ is the number of single (uncoupled) susceptible individuals, and $I$ is the number of single infected individuals. $SS$ is the number of susceptible-susceptible couples, $SI$ is the number of serodiscordant (susceptible-infected) couples, and $II$ is the number of concordant positive (infected-infected) couples. Model 1 includes extra-partnership contact (with both uncoupled individuals and individuals in other partnerships) whereas model 2 only considers within-couple transmission. Models 3 and 4 assume instantaneous partnership formation and thus consist of only the three partnered states: $SS$, $SI$, and $II$. Parallel to model 1 and 2, model 3 includes extra-partnership contact (now only with individuals in other partnerships, since uncoupled individuals don't exist in this model) and model 4 only considers within-couple transmission.

In contrast, models 5 and 6 are not explicitly structured.  Model 5 is an implicit serial monogamy model based on the epidemiological model used by Shirreff \etal\ \cite{shirreff_transmission_2011}. It is actually a random mixing model that consist of only two states, $S$ and $I$, and does not consider explicit partnership dynamics. However, to simulate the effect of instantaneous partnership formation, it uses an adjusted transmission rate that is derived from approximated basic reproduction number of a serial monogamy model \cite{hollingsworth_hiv1_2008}. Finally, model 6 is a simple random-mixing model.

The base model (i.e. model 1) for the first four models is an extension of Champredon text{et al.}'s model \cite{champredon_hiv_2013}. Individuals in single compartment acquire a partner at a rate, $\rho$, and partnerships dissolve at a rate, $c$. Infected individuals in a discordant partnership infect susceptible partner at a rate $\beta$ (within-couple transmission rate) and susceptible individuals outside the partnership at a rate $c_e$ (extra-couple transmission rate). Likewise, a single infected individual can infect any susceptible individuals at a rate $c_u$ through uncoupled mixing. Extra-couple and uncoupled transmission are modeled in a same way as Champredon text{et al.}'s model. All the details have been adapted to a multi-strain scenario. Model 2, 3, and 4 are derived from the base model by removing epidemiological details (partnership formation and uncoupled/extra-couple contact). Model details are explained in the appendix.


\subsection*{Latin hypercube sampling}

Despite considerable effort by many researchers \cite{hollingsworth_hiv1_2008,champredon_hiv_2013}, the parameters determining the rate and structure of sexual partnership change and contact are still very uncertain; this led Champredon \etal\ \cite{champredon_hiv_2013} to adopt a Latin hypercube sampling (LHS) strategy \cite{blower_drugs_1991} that evaluates model outcomes over a range of parameter values. In order to make sure that our comparisons among models apply across the entire space of reasonable parameter values, and in order to evaluate the differential sensitivity of different model structures to parameter values, we follow a similar protocol and perform LHS over a parameter set including both the transmission and duration parameters ($\beta_P$, $D_P$, $\beta_D$, $D_D$) and contact/partnership parameters ($\rho$, $c$, $c_u/c_w$, and $c_e/c_w$). We do not allow for uncertainties in parameters that are directly related to the evolutionary process ($\beta_{max}$, $\beta_{50}$, $\beta_k$, $D_{max}$, $D_{50}$, $D_k$, $\sigma_M$), using Shirreff \etal's point estimates throughout \cite{shirreff_transmission_2011}.

Latin hypercube sampling is done as in Champredon \etal\ \cite{champredon_hiv_2013}. For each parameter, $z$, its range is divided into $N = 1000$ equal intervals on a log scale:

\begin{equation}
z_i = \exp\left(\log(z_{min}) + [\log(z_{max}) - \log(z_{min})] \frac{i-1}{N-1}\right) \qquad i = 1,...,N.
\end{equation}
For simplicity, we assume that all parameters are uniformly distributed on the log scale. Following the vectorization of a parameter range, a matrix is constructed so that each column contains a vector of a parameter series which it represents ($z_1$,...$z_N$). Then, each column is replaced with a random permutation series of itself. Now, each row contains a different parameter set that is used for each simulation run.

Table~\ref{table:parmsTable} gives the ranges of the model parameters used for LHS. Parameter ranges regarding contact and partnership dynamics ($\rho$, $c$, and $c_e/c_w$) are taken from Champredon \etal\ \cite{champredon_hiv_2013}, whereas those regarding infection ($\beta_P$, $D_P$, $\beta_D$, and $D_D$) are taken from Hollingsworth \etal\ \cite{hollingsworth_hiv1_2008}. The remaining parameters are taken from Shirreff \etal\ \cite{shirreff_transmission_2011}.

%Tables need to be placed after the first paragraph in which they are cited.

\begin{table}[h!]
\caption{Parameter ranges/values.  Note that $c$ and $\rho$ values are doubled from Champredon \etal\  because we keep track of individuals, while they keep track of couples. Starred (*) parameters (used in Figure~\ref{fig:panel3}) are from \cite{shirreff_transmission_2011}.}
\centering
\begin{tabular}{c p{2in} c l}
\hline 
Notation & Description & Range/Value & Source\\
\hline % inserts single horizontal line
$\rho$ & Partnership formation rate & 1/10-2/5 & \cite{champredon_hiv_2013} \\
$c$ & Partnership dissolution rate & 1/15-1/5 (1.25*) & \cite{champredon_hiv_2013} \\
$c_u/c_w$ & Relative contact rate for uncoupled transmission & 1/5-5 & Assumption \\
$c_e/c_w$ & Relative contact rate extra-couple & 0.01-1 & \cite{champredon_hiv_2013} \\
$\beta_P$ & Rate of transmission during primary infection & 1.31-5.09 (2.76*) & \cite{hollingsworth_hiv1_2008} \\
$\beta_D$ & Rate of transmission during high transmission disease stage & 0.413-1.28 (0.76*) & \cite{hollingsworth_hiv1_2008} \\
$D_P$ & Duration of primary infection & 1.23/12-6/12 (0.25*) & \cite{hollingsworth_hiv1_2008,shirreff_transmission_2011} \\
$D_D$ & Duration of high transmission disease stage & 4.81/12-14/12 (0.75*) & \cite{hollingsworth_hiv1_2008} \\
$\tsub{\beta}{max}$ & Maximum rate of transmission during asymptomatic stage & 0.317 & \cite{shirreff_transmission_2011} \\
$\beta_{50}$ & SPVL at which infectiousness is half maximum & 13938 & \cite{shirreff_transmission_2011} \\
$\beta_k$ & Hill coefficient: steepness of increase in infectiousness as a function of SPVL & 1.02 & \cite{shirreff_transmission_2011} \\
$\tsub{D}{max}$ & Duration of primary infection & 25.4 & \cite{shirreff_transmission_2011} \\
$D_{50}$ & SPVL at which duration of asymptomatic infection is half maximum & 3058 & \cite{shirreff_transmission_2011} \\
$D_{k}$ & Hill coefficient: steepness of decrease in duration as a function of SPVL & 0.41 & \cite{shirreff_transmission_2011} \\
$\sigma_M$ & Mutation standard deviation of $\log_{10}$ SPVL & 0.12 & \cite{shirreff_transmission_2011} \\
$\tsub{\alpha}{min}$ & Minimum $\log_{10}$ SPVL & 2 & \cite{shirreff_transmission_2011}\\
$\tsub{\alpha}{max}$ & Maximum $\log_{10}$ SPVL & 7 & \cite{shirreff_transmission_2011}\\
$n$ & Number of strains & 21 (51*) & Assumption\\
\hline
\end{tabular}
\label{table:parmsTable}
\end{table}

The one completely new parameter in our model, the ratio of uncoupled to within-couple transmission $c_u/c_w$, is needed to more flexibly contrast uncoupled and extra-couple transmission dynamics within multi-strain models (Appendix); we need to pick a reasonable range for it. Champredon \etal\ \cite{champredon_hiv_2013} assume that the effective within-couple contact rate and effective uncoupled contact rate have the same range of 0.05 - 0.25.  Given Champredon \etal's parameter range, the possible maximum and minimum values of $c_u/c_w$ are 5 and 1/5. Therefore, we use 1/5-5 as the range for the parameter $c_u/c_w$. Although this adds more uncertainty to the parameter $c_u$ --- Champredon \etal's range implies a 5-fold difference whereas ours gives a 25-fold difference --- as there is not much known about the uncoupled transmission rate, we consider the wider range appropriate.


\subsection*{Simulation runs}

One of the most difficult parts of model comparison is finding parameter sets that are commensurate against many different model structures. For the most part, our models are too complex to easily derive analytical correspondences among them. Given a numerical criterion, such as $r$ (initial exponential growth rate) or ${\cal R}_0$ (intrinsic reproductive number), we can adjust one or more parameters by brute force to ensure that all of the models match according to that criterion. While ${\cal R}_0$ is often considered the most fundamental property of an epidemic, and might thus seem to be a natural goal, here we focus on matching the initial growth rate $r$ for several reasons. First, our primary interest is in the transient evolutionary dynamics of virulence, which are more strongly affected by $r$ than ${\cal R}_0$. Second, $r$ is in general more directly observable in real epidemics; $r$ can be estimated by simply fitting an exponential curve to the initial incidence or prevalence curves \cite{ma_estimating_2014}, while ${\cal R}_0$ typically requires either (1) knowledge of \emph{all} epidemic parameters or (2) relatively sophisticated back-calculation based on $r$ and knowledge of the serial interval or generation interval of the disease \cite{wallinga_how_2007}. Thus, we scale a parameter so that every run has the same initial exponential growth rate in the incidence.

In order to allow for all models to have equal initial exponential growth rate, $r$, we need to pick a parameter, $s$, so that $\lim_{s\to 0} r(s) = 0$ and $\lim_{s\to\infty}  r(s) = \infty$. As adjusting either partnership change rate (i.e. partnership formation and dissolution rate) or transmission rate does not fulfill this requirement for certain models, we decided to scale partnership change rate and dissolution rate by the same factor of $\gamma$: $\beta_{adj} =  \gamma \beta_{base}$, $c_{adj} = \gamma c_{base}$, $\rho_{adj} = \gamma \rho_{base}$. Since transmission rate is adjusted by the scale of $\gamma$, uncoupled and extra-couple transmission rates are adjusted as well. For models 3, 4, and 5, all of which assume instantaneous partnership (and hence do not track single individuals), only the transmission rate and partnership dissolution rate (in this case equivalent to the partnership change rate) are adjusted.

We run each model for each of 1000 parameter sets chosen by Latin hypercube sampling, with fixed starting conditions
of mean \Lspvl\ of 3 and epidemic size of $10^{-4}$. After each run, initial exponential growth rate is calculated. Then, parameters are scaled so that the initial exponential growth rate is scaled to 0.04, which is approximately equal to that of Shirreff et al's model.

For each model we derive the following summary statistics: peak virulence, peak time, equilibrium virulence, and relative peak virulence. The transient phase of an epidemic is often characterized by high virulence, and we define peak virulence as the maximum virulence during this phase. It is simply calculated by taking the maximum value from the virulence trajectory, and peak time is the time at which the maximum value is reached. Once the epidemic enters the endemic phase, evolution of virulence stabilizes and reaches equilibrium. Equilibrium virulence is calculated by taking the mean virulence at 4000 years. Although most simulations reach equilibrium much earlier, we set our time horizon at a much later date as some simulation runs have slow rate of evolution depending on the parameter set and model assumptions.

We focus on these statistics for the following two reasons. First of all, knowing the possible ranges for the peak virulence allows us to estimate the worst-case scenario for the HIV and other sexually transmitted disease epidemics. Pathogens may already have evolved towards high virulence during the early stages of an epidemic, by the time it is observed by public health authorities. Understanding how virulent a pathogen can evolve before an epidemic begins can be helpful for controlling the disease. Furthermore, knowing the initial virulence, peak virulence, timing of the peak virulence, and equilibrium virulence provide sufficient detail to identify the shape of the virulence trajectory. During an epidemic outbreak, it is difficult to observe virulence evolution. Specifically, in the case of HIV and other sexually transmitted diseases, slow evolutionary time-scale makes observing changes in the mean virulence even more challenging. Knowing the ranges of these statistics can help real-time virulence evolution prediction during an epidemic less troublesome.

% Results and Discussion can be combined.
\section*{Results}

Our simplifications of Shirreff \etal's model \cite{shirreff_transmission_2011} reproduce its qualitative behaviour --- in particular, its predictions of virulence dynamics --- reasonably well. As $r$ decreases from 0.084 to 0.42 (the latter value matching the initial rate of increase in prevalence in Shirreff \etal's full model) the initial trajectory of increasing virulence brackets the rate from the original model (Fig.~\ref{fig:panel3}a). However, our model produces lower peak virulence ($\approx 4.3$ vs. $\approx 4.6$ \Lspvl) 
and equilibrium virulence $\approx 4.25$ vs. $\approx 4.5$ \Lspvl) than Shirreff's, even for matching initial incidence trajectories (i.e., $r=0.042~\textrm{year}^{-1}$).

Changing the initial infectious density ($I(0)$), while it produces the expected changes in the initial epidemic trajectory (Supplementary material), has little effect on the virulence trajectory, making the virulence peaks slightly later and larger as $I(0)$ decreases. Decreasing $I(0)$ allows a longer epidemic phase before the transition to endemic dynamics (Fig.~\ref{fig:panel3}b). Decreasing the initial virulence similarly but more strongly leads to progressively later, larger peaks in virulence (Fig.~\ref{fig:panel3}c).

Across the entire range of parameters covered by the LHS analysis, all of the classes of models we considered produce qualitatively similar virulence trajectories (Fig.~\ref{fig:virtraj}). Although the speed of virulence evolution varies, leading to wide variation in the peak virulence (means ranging from approximately 3.75 to 4.5 \Lspvl), virulence peaks in all models between 200 and 300 years.

\begin{figure}[!h]
\caption{{\bf Envelopes of virulence trajectories under all models.}
All models were run until $t=4000~\textrm{years}$; truncated series are shown here.}
\label{fig:virtraj}
\end{figure}

Our chosen summary statistics (peak time, peak virulence, equilibrium
virulence, and relative peak virulence) all vary considerably across models
\ref{fig:unidist}.
We first consider the models of intermediate realism (implicit,
instantaneous-switching with and without extra-pair contact, and
pair formation without extra-pair contact). Some parameter
sets for these models lead to low equilibrium virulence ($\approx 2.5 \log_{10}$ SPVL);
these same sets lead to correspondingly low
peak virulence ($< 3.5 \log_{10}$ SPVL) and early peak times (before 200 years), 
but high relative peaks ($>1.3$)
(Fig.~\ref{fig:pairplot}, leftmost column) because the equilibrium virulence is low.
At the opposite extreme, parameter sets that produce high equilibrium virulence 
also produce late peaks ($> 200 \text{ years}$), 
high peak virulence, and low relative peaks ($\approx 1.05$).
The pair-formation without extra-pair contact and implicit models
occasionally have parameter sets that select for such low virulence across
the board that they never exceed their initial virulence, leading to a tail
of peak times near zero.

\begin{figure}[!h]
\caption{{\bf Univariate distributions of summary statistics.}
The distribution of equilibrium virulence for the random mixing model is very narrow, and has been replaced by a point in order to preserve the vertical axis scaling.}
\label{fig:unidist}
\end{figure}

\begin{figure}[!h]
\caption{{\bf Pairs plot: bivariate relationships among summary statistics for each model structure.}
Dashed line in equilibrium vs. peak virulence plot shows 1:1 line.}
\label{fig:pairplot}
\end{figure}

The most striking aspect of the univariate comparisons in
Fig.~\ref{fig:unidist}, (and the bivariate comparisons in
Fig.~\ref{fig:pairplot}) is the match between the output of the
least (random mixing) and the most complex (pair formation with
extra-pair contact) models. The random-mixing model has lower
variability, because it is unaffected by uncertainty in pair formation
and extra-pair contact parameters, but otherwise the virulence
dynamics of these two extreme models are remarkably similar.
This phenomenon is driven by the strength of extra-pair contact in the
model with explicit pair formation and extra-pair contact 
(``pairform+epc'' in Figures~\ref{fig:virtraj}-\ref{fig:plot_sens}). When individuals spend time uncoupled between
partnerships, and when these single individuals can transmit disease
to coupled individuals, the resulting unstructured mixing overwhelms
the effect of structured mixing within couples, leading to mixing
that is effectively close to random.

\begin{figure}[!h]
\caption{{\bf Sensitivity plot.}
For each parameter in the Latin hypercube sample and each summary statistic, shows the distribution (points) and trend (smooth line) of the summary statistic as a function of the \emph{unscaled} parameter value, i.e. prior to adjusting the parameters to achieve the standard initial epidemic growth rate.}
\label{fig:plot_sens}
\end{figure}

Plotting the bivariate result distributions (Fig.~\ref{fig:pairplot})
shows that most of the relationships among the summary statistics
are monotonic, except those involving the relative peak virulence
(bottom row). Changes in parameters that increase the equilibrium
virulence initially increase the peak virulence even more, so
that the relative peak virulence increases as well, but beyond
an equilibrium virulence of about 2.5 \Lspvl\ the peak virulence
increases slower than the equilibrium virulence, leading to a
decrease in the relative peak virulence.

The bivariate relationships also help distinguish the results of 
different models with similar univariate distributions. While the
relationship between equilibrium virulence and peak time is
similar for all model structures (top left panel), the other
relationships are more separate. In particular, the implicit
and pair-formation (without extra-pair contact) are very similar
to each other, but distinct from the other models. We still do
not have a convincing explanation for this distinction; we
would have expected the implicit model to be most similar to the
the instantaneous-switching model without extra-pair contact,
which most closely matches its derivation. However, we note
that the implicit model derivation is based on defining
the force of infection to match a scaled version of ${\cal R}_0$,
and as such would be expected to match the equilibrium behaviour
but not necessarily the epidemic-phase behaviour of a model
with explicit partnership dynamics.

Finally, the sensitivity plots (Fig.~\ref{fig:plot_sens}) show the effects 
of each parameter on the summary statistics. In almost every case the
effects of the parameters are monotonic; note that
the plot shows the effects of the \emph{unscaled} parameters, i.e.
before they have been adjusted to achieve a standard initial epidemic
growth rate.
Increases in the transmission rates ($\beta_P$, $\beta_D$)
and durations ($D_P$, $D_D$) in the primary and disease stages generally
decrease the equilibrium virulence, peak virulence, and peak time,
although the random and pair-formation+extra-pair contact
models have high, relatively
constant values with respect to these parameters. 

The base contact rate increases virulence and peak time in almost all
cases, although the pair-formation+extra-pair contact
model is again relatively insensitive.
The ratio of extra-pair to within-pair contact ($c_e/c_w$) affects
virulence in the instantaneous-switching+epc model, but not the pair-formation+epc
model (probably because the uncoupled individuals present in the pair-formation+epc
model make extra-pair contact by coupled individuals less important).
Surprisingly, neither of the pair-formation models is particularly sensitive to the 
rate of partnership formation ($\rho$); the rate of uncoupled contact
increases virulence and peak time in the pair-formation+epc model, 
which is the only model to which it applies.

\section*{Discussion}

All models must simplify the world.  Many constraints --- such as data
availability, computation time, or code complexity --- drive the need
for parsimony, with different constraints applying in different
contexts. The critical question that modelers must ask is whether the
simplified model gives adequate answers, or whether the
simplifications have led to qualitatively or quantitatively incorrect
conclusions.  This question is especially important for modelers who
are hoping that their conclusions will guide management decisions.

In the particular example of HIV virulence eco-evolutionary dynamics
and epidemiological, we reach the slightly ironic conclusion that the
effort put into building a more realistic model essentially cancels
out, leaving us in the same position as if we had ignored the problem
of epidemiological complexity entirely and used a naive random-mixing
contact model. Herbeck \etal\ \cite{herbeck_hiv_2014} build a
network model of partnerships, but they set the ``partnership''
duration to 1 day --- very unrealistic in epidemic terms, but perhaps
actually more true to real-world HIV epidemiological dynamics than a
model with realistic partnership durations that neglects extra-pair
contact \cite{herbeck2016evolution}. Adding still further realism
might push the results farther away from the random-mixing model. For
example, our model forms partnerships randomly, and assumes that
extra-pair contact is randomly mixing across the population;
one could instead implement extra-pair contact as arising from
multiple concurrent partnerships (some, such as contact with sex
workers, of very short duration), or more structured partnership
formation (by age, ethnicity, or behaviour group). The effects of
other realistic complications such as explicit modeling of two
sexes (both in contact structure and differential transmission
probabilities), temporal and spatial variation in epidemic processes,
or presence of genetic variation in hosts are harder to predict.

Parameterization is one of the biggest challenges of epidemiological
modeling. In addition to following Champredon \etal\ \cite{champredon_hiv_2013} 
by doing Latin hypercube
sampling across a wide range of epidemiological parameters, we 
calibrated each set of parameters to the same initial epidemic
growth rate, chosen to match the results of previous models
\cite{shirreff_transmission_2011}.  The previous models drew their
parameters from previous cohort studies from the 1990s
\cite{wawer2005rates,hollingsworth_hiv1_2008}
rather than doing any explicit calibration to epidemic curves,
but they give reasonable order-of-magnitude
growth rates ($\approx 0.04~\textrm{year}^{-1}$)
for the early stages of the HIV epidemic (although considerably
lower than estimates of $\approx 0.07-0.1~\textrm{year}^{-1}$
based on population genetic reconstructions \cite{faria_early_2014}).
However, our reason for calibrating was not to match any
specific observed epidemic, but rather to make sure that
we were making meaningful comparisons across a range of
models with radically different epidemiological structures, and
hence involving different interpretations of the same quantitative
parameters.  For example, in models with instantaneous switching the
partnership dissolution rate $c$ is identical to the partnership
formation rate; in models with explicit partnership formation,
the partnership formation rate is also $c$ at equilibrium,
but might vary over the course of an epidemic.
It is not obvious whether models with equal parameters but
different structures should be directly compared; calibration
solves this problem.

More generally, any model that wants to be
taken seriously for management and forecasting purposes should
be calibrated to \emph{all} available data, using informative
priors to incorporate both realistic distributions of uncertainty
for all parameters from independent measurements \cite{elderd_uncertainty_2006}
and calibration from population-level observations of epidemic
trajectories. For practical (rather than exploratory) purposes,
such a procedure would also be an improvement on the common --- although not universal --- %
practice of assessing uncertainty over uniform ranges, rather than
distributions that allow more continuous variation in support over
the range of a parameter.

Researchers have documented that HIV virulence and set-point viral
load are changing, on time scales comparable to those portrayed here
(e.g., compare Fig.~\ref{fig:virtraj} to Herbeck \text{et al}'s
estimated rate of change of 1.3 \Lspvl\ per century [95\% CI -0.1 to
  3] \cite{herbeck_is_2012}), and have begun to build relatively realistic models that
attempt to describe how interventions such as mass antiretroviral
therapy (ART) can be expected to change the trajectory of virulence
evolution \cite{payne_impact_2014,herbeck2016evolution}.  While these
efforts are well-intentioned, we caution that epidemiological and
other structural details that are currently omitted from these models
could significantly change their conclusions.

\section*{Conclusion}

CO\textsubscript{2} Maecenas convallis mauris sit amet sem ultrices gravida. Etiam eget sapien nibh. Sed ac ipsum eget enim egestas ullamcorper nec euismod ligula. Curabitur fringilla pulvinar lectus consectetur pellentesque. Quisque augue sem, tincidunt sit amet feugiat eget, ullamcorper sed velit. 

Sed non aliquet felis. Lorem ipsum dolor sit amet, consectetur adipiscing elit. Mauris commodo justo ac dui pretium imperdiet. Sed suscipit iaculis mi at feugiat. Ut neque ipsum, luctus id lacus ut, laoreet scelerisque urna. Phasellus venenatis, tortor nec vestibulum mattis, massa tortor interdum felis, nec pellentesque metus tortor nec nisl. Ut ornare mauris tellus, vel dapibus arcu suscipit sed. Nam condimentum sem eget mollis euismod. Nullam dui urna, gravida venenatis dui et, tincidunt sodales ex. Nunc est dui, sodales sed mauris nec, auctor sagittis leo. Aliquam tincidunt, ex in facilisis elementum, libero lectus luctus est, non vulputate nisl augue at dolor. For more information, see \nameref{S1_Appendix}.

\section*{Supporting Information}

% Include only the SI item label in the paragraph heading. Use the \nameref{label} command to cite SI items in the text.
\paragraph*{S1 Fig.}
\label{S1_Fig}
{\bf Bold the title sentence.} Add descriptive text after the title of the item (optional).

\paragraph*{S2 Fig.}
\label{S2_Fig}
{\bf Lorem Ipsum.} Maecenas convallis mauris sit amet sem ultrices gravida. Etiam eget sapien nibh. Sed ac ipsum eget enim egestas ullamcorper nec euismod ligula. Curabitur fringilla pulvinar lectus consectetur pellentesque.

\paragraph*{S1 File.}
\label{S1_File}
{\bf Lorem Ipsum.}  Maecenas convallis mauris sit amet sem ultrices gravida. Etiam eget sapien nibh. Sed ac ipsum eget enim egestas ullamcorper nec euismod ligula. Curabitur fringilla pulvinar lectus consectetur pellentesque.

\paragraph*{S1 Video.}
\label{S1_Video}
{\bf Lorem Ipsum.}  Maecenas convallis mauris sit amet sem ultrices gravida. Etiam eget sapien nibh. Sed ac ipsum eget enim egestas ullamcorper nec euismod ligula. Curabitur fringilla pulvinar lectus consectetur pellentesque.

\paragraph*{S1 Appendix.}
\label{S1_Appendix}
{\bf Lorem Ipsum.} Maecenas convallis mauris sit amet sem ultrices gravida. Etiam eget sapien nibh. Sed ac ipsum eget enim egestas ullamcorper nec euismod ligula. Curabitur fringilla pulvinar lectus consectetur pellentesque.

\paragraph*{S1 Table.}
\label{S1_Table}
{\bf Lorem Ipsum.} Maecenas convallis mauris sit amet sem ultrices gravida. Etiam eget sapien nibh. Sed ac ipsum eget enim egestas ullamcorper nec euismod ligula. Curabitur fringilla pulvinar lectus consectetur pellentesque.

\section*{Acknowledgments}
NSERC; C. Fraser

\nolinenumbers

% Either type in your references using
% \begin{thebibliography}{}
% \bibitem{}
% Text
% \end{thebibliography}
%
% or
%
% Compile your BiBTeX database using our plos2015.bst
% style file and paste the contents of your .bbl file
% here.
% 

\bibliography{virulence}

\end{document}

