% ## http://journals.plos.org/ploscompbiol/s/submit-now
% Template for PLoS
% Version 3.2 March 2016
%
% % % % % % % % % % % % % % % % % % % % % %
%
% -- IMPORTANT NOTE
%
% This template contains comments intended 
% to minimize problems and delays during our production 
% process. Please follow the template instructions
% whenever possible.
%
% % % % % % % % % % % % % % % % % % % % % % % 
%
% Once your paper is accepted for publication, 
% PLEASE REMOVE ALL TRACKED CHANGES in this file 
% and leave only the final text of your manuscript. 
% PLOS recommends the use of latexdiff to track changes during review, as this will help to maintain a clean tex file.
% Visit https://www.ctan.org/pkg/latexdiff?lang=en for info or contact us at latex@plos.org.
%
%
% There are no restrictions on package use within the LaTeX files except that 
% no packages listed in the template may be deleted.
%
% Please do not include colors or graphics in the text.
%
% The manuscript LaTeX source should be contained within a single file (do not use \input, \externaldocument, or similar commands).
%
% % % % % % % % % % % % % % % % % % % % % % %
%
% -- FIGURES AND TABLES
%
% Please include tables/figure captions directly after the paragraph where they are first cited in the text.
%
% DO NOT INCLUDE GRAPHICS IN YOUR MANUSCRIPT
% - Figures should be uploaded separately from your manuscript file. 
% - Figures generated using LaTeX should be extracted and removed from the PDF before submission. 
% - Figures containing multiple panels/subfigures must be combined into one image file before submission.
% For figure citations, please use "Fig" instead of "Figure".
% See http://journals.plos.org/plosone/s/figures for PLOS figure guidelines.
%
% Tables should be cell-based and may not contain:
% - tabs/spacing/line breaks within cells to alter layout or alignment
% - vertically-merged cells (no tabular environments within tabular environments, do not use \multirow)
% - colors, shading, or graphic objects
% See http://journals.plos.org/plosone/s/tables for table guidelines.
%
% For tables that exceed the width of the text column, use the adjustwidth environment as illustrated in the example table in text below.
%
% % % % % % % % % % % % % % % % % % % % % % % %
%
% -- EQUATIONS, MATH SYMBOLS, SUBSCRIPTS, AND SUPERSCRIPTS
%
% IMPORTANT
% Below are a few tips to help format your equations and other special characters according to our specifications. For more tips to help reduce the possibility of formatting errors during conversion, please see our LaTeX guidelines at http://journals.plos.org/plosone/s/latex
%
% For inline equations, please be sure to include all portions of an equation in the math environment.  For example, x$^2$ is incorrect; this should be formatted as $x^2$ (or $\mathrm{x}^2$ if the romanized font is desired).
%
% Do not include text that is not math in the math environment. For example, CO2 should be written as CO\textsubscript{2} instead of CO$_2$.
%
% Please add line breaks to long display equations when possible in order to fit size of the column. 
%
% For inline equations, please do not include punctuation (commas, etc) within the math environment unless this is part of the equation.
%
% When adding superscript or subscripts outside of brackets/braces, please group using {}.  For example, change "[U(D,E,\gamma)]^2" to "{[U(D,E,\gamma)]}^2". 
%
% Do not use \cal for caligraphic font.  Instead, use \mathcal{}
%
% % % % % % % % % % % % % % % % % % % % % % % % 
%
% Please contact latex@plos.org with any questions.
%
% % % % % % % % % % % % % % % % % % % % % % % %

\documentclass[10pt,letterpaper]{article}
\usepackage[top=0.85in,left=2.75in,footskip=0.75in]{geometry}

% Use adjustwidth environment to exceed column width (see example table in text)
\usepackage{changepage}

% Use Unicode characters when possible
\usepackage[utf8x]{inputenc}

% textcomp package and marvosym package for additional characters
\usepackage{textcomp,marvosym}

% fixltx2e package for \textsubscript
\usepackage{fixltx2e}

% amsmath and amssymb packages, useful for mathematical formulas and symbols
\usepackage{amsmath,amssymb}

% cite package, to clean up citations in the main text. Do not remove.
\usepackage{cite}

% Use nameref to cite supporting information files (see Supporting Information section for more info)
\usepackage{nameref,hyperref}

% line numbers
\usepackage[right]{lineno}

% ligatures disabled
\usepackage{microtype}
\DisableLigatures[f]{encoding = *, family = * }

% Remove comment for double spacing
\usepackage{setspace} 
\doublespacing

% Text layout
\raggedright
\setlength{\parindent}{0.5cm}
\textwidth 5.25in 
\textheight 8.75in

% Bold the 'Figure #' in the caption and separate it from the title/caption with a period
% Captions will be left justified
\usepackage[aboveskip=1pt,labelfont=bf,labelsep=period,justification=raggedright,singlelinecheck=off]{caption}
\renewcommand{\figurename}{Fig}

% Use the PLoS provided BiBTeX style
\bibliographystyle{plos2015}

% Remove brackets from numbering in List of References
\makeatletter
\renewcommand{\@biblabel}[1]{\quad#1.}
\makeatother

% Leave date blank
\date{}

% Header and Footer with logo
\usepackage{lastpage,fancyhdr,graphicx}
\usepackage{epstopdf}
\pagestyle{myheadings}
\pagestyle{fancy}
\fancyhf{}
\setlength{\headheight}{27.023pt}
\lhead{\includegraphics[width=2.0in]{PLOS-submission.eps}}
\rfoot{\thepage/\pageref{LastPage}}
\renewcommand{\footrule}{\hrule height 2pt \vspace{2mm}}
\fancyheadoffset[L]{2.25in}
\fancyfootoffset[L]{2.25in}
\lfoot{\sf PLOS}

%% Include all macros below
\newcommand{\khalf}{\left(\frac{1}{2}\right)^{\delta_{ij}}}  % (1/2)^kronecker
\newcommand{\kkhalf}{\left(\frac{1}{2}\right)^{\delta_{ij} \delta_{kl}}}  % (1/2)^(kronecker * kronecker)
\newcommand{\Lspvl}{$\log_{10}$ SPVL}
\newcommand{\rzero}{{\mathcal R}_0}
\newcommand{\etal}{\textit{et al.}}
\newcommand{\tsub}[2]{#1_{{\textrm{\tiny #2}}}}
\newcommand{\PF}{\textrm{PF}}
\newcommand{\DS}{\textrm{DS}}
\newcommand{\WT}{\textrm{WT}}
\newcommand{\ET}{\textrm{ET}}
\newcommand{\DM}{\textrm{DM}}

\newcommand{\todo}[1]{\textbf{#1}}

%% END MACROS SECTION


\begin{document}
\vspace*{0.2in}

% Title must be 250 characters or less.
\begin{flushleft}
{\Large
\textbf\newline{Effects of Epidemiological Structure on the Transient Evolution of HIV Virulence} % Please use "title case" (capitalize all terms in the title except conjunctions, prepositions, and articles).
}
%% short title: Epidemiological Structure and HIV Virulence Evolution
%% (53/70 characters)
\newline
% Insert author names, affiliations and corresponding author email (do not include titles, positions, or degrees).
\\
Sang Woo Park\textsuperscript{1}
Benjamin M. Bolker\textsuperscript{1,2,3,*}
\\
\bigskip
\textbf{1} Department of Mathematics \& Statistics,  McMaster University, Hamilton, Ontario, Canada
\\
\textbf{2} Department of Biology,  McMaster University, Hamilton, Ontario, Canada
\\
\textbf{3} Institute for Infectious Disease Research,  McMaster University, Hamilton, Ontario, Canada
\\
\bigskip

% Insert additional author notes using the symbols described below. Insert symbol callouts after author names as necessary.
% 
% Remove or comment out the author notes below if they aren't used.
%
% Primary Equal Contribution Note
%\Yinyang These authors contributed equally to this work.

% Additional Equal Contribution Note
% Also use this double-dagger symbol for special authorship notes, such as senior authorship.
%\ddag These authors also contributed equally to this work.

% Current address notes
%\textcurrency Current Address: Dept/Program/Center, Institution Name, City, State, Country % change symbol to "\textcurrency a" if more than one current address note
% \textcurrency b Insert second current address 
% \textcurrency c Insert third current address

% Deceased author note
%\dag Deceased

% Group/Consortium Author Note
%\textpilcrow Membership list can be found in the Acknowledgments section.

% Use the asterisk to denote corresponding authorship and provide email address in note below.
* bolker@mcmaster.ca

\end{flushleft}
% Please keep the abstract below 300 words
% current is 282
\section*{Abstract}
The evolutionary dynamics of parasite virulence change
in important ways over the
course of an emerging epidemic.
Changes in the fitness landscape
generally select for higher virulence early in an
epidemic; however, quantitative outcomes
may depend sensitively on epidemiological details and the structure of
mathematical models used to portray them.  Fraser \emph{et al.} 
have proposed a model for the eco-evolutionary
dynamics of HIV that captures
the tradeoffs between transmission and virulence (mediated by
set-point viral load, SPVL) and their heritability between
hosts. However, these models use implicit
representations of the transmission process that drastically simplify the
partnership dynamics that previous research has found to be critical
in driving epidemics of sexually transmitted diseases.  
Our models combine HIV virulence tradeoffs with a range of
epidemiological structures, modeling partnership formation and
dissolution and allowing for individuals to transmit disease outside
of partnerships. We assess summary statistics such as the peak virulence
(corresponding to the minimum expected time of progression to AIDS) across all
models for a range of 
partnership dynamic parameters 
applicable to the HIV epidemic in sub-Saharan Africa.
Although virulence trajectories are broadly similar
across model structures, the timing and magnitude of the 
minimum expected time to progression vary
considerably.
Models of intermediate
complexity as used by Fraser \emph{et al.} \todo{why not shirreff et al?} 
predicted lower slower progression/lower virulence (a minimum of 15 years to
progress to AIDS) compared to both more realistic models
and simple random-mixing models with no partnership structure
at all (both with a minimum of $\approx$ 7.25 years to progress to AIDS).
In this range of models, the simplest random-mixing structure best
approximates the most realistic model; this
surprising outcome occurs because the dominance of extra-pair
contact in the realistic model tends to swamp the effects of
partnership structure.

% Please keep the Author Summary between 150 and 200 words
% Use first person. PLOS ONE authors please skip this step. 
% Author Summary not valid for PLOS ONE submissions.   
\section*{Author Summary}

%% 162 words

Pathogens such as HIV can evolve rapidly in response to changes in
their environments; such changes include both increases in disease
prevalence and disease virulence over the course of an epidemic, or
decreases in both after treatment interventions. While researchers
have successfully used computational models to explore these
evolutionary dynamics, these models often neglect details such as the
formation and dissolution of sexual partnerships; other research has
shown that these processes can strongly affect epidemic outcomes. We
built and compared models that used different methods to model both
partnership dynamics and sexual contact outside of stable
partnerships. Models of intermediate complexity predicted much lower
virulence over the course of the epidemic (a minimum of 15 years to
progress to AIDS) compared to both more realistic models and simple
random-mixing models with no partnership structure at all (both
approx. 7.25 years to progress to AIDS); sexual contact outside of
stable partnerships tended to wash out the effects of epidemiological
structure. The large differences in evolutionary dynamics among
different epidemiological models suggests that researchers trying to
predict the evolution of pathogens should proceed with caution.

\linenumbers

% Use "Eq" instead of "Equation" for equation citations.
\section*{Introduction}

The evolution of pathogen virulence has both 
theoretical and, potentially, practical
importance. In general, evolutionary theory suggests that
disease strains that can reproduce more --- where reproduction
is defined here as the amount of \emph{between-host} transmission, or
the number of new hosts infected --- will increase in prevalence.
Pathogens can increase their net reproduction rate either
by increasing their transmission rate, 
the rate (per infected host) at which they
infect new hosts, or by decreasing their clearance or disease-induced
mortality rate, the rate
at which hosts recover or die from disease.
The \emph{trade-off theory} \cite{alizon_virulence_2009} postulates that
the transmission and disease-induced mortality rate are both linked to the rate at which
the pathogen exploits host resources for within-host reproduction, 
and that pathogen
evolution will thus strike a balance between the
pathogen's rate of transmission
to new hosts and its rate of killing its host (or of provoking
the host's immune system to eliminate it).
Some biologists have criticized the tradeoff theory
\cite{EbertBull2003,alizon_adaptive_2015}, but others have
successfully applied it to a variety of host-pathogen systems \cite{Dwyer+1990,mackinnon1999genetic,jensen2006empirical,deroode2008virulence}.
Fraser \etal\ have applied these ideas in a particularly
interesting way 
by showing that HIV appears to satisfy the prerequisites of
the tradeoff theory: in studies of discordant couples (i.e. long-term
sexual partnerships with one infected and one uninfected partner), HIV
virulence as measured by the rate of progression to AIDS was both
heritable and covaried with the set-point viral load (i.e., the
characteristic virus load measured in blood during the intermediate
stage of infection), which in turn predicted the probability of
transmission
\cite{Fraser+2007,fraser_virulence_2014}. Subsequent studies
\cite{shirreff_transmission_2011,herbeck_hiv_2014} used these data to
parameterize mechanistic models of HIV virulence evolution, suggesting
that HIV invading a novel population would initially evolve increased
virulence, peaking after approximately 100-200 years and then declining
slightly to a long-stable virulence level.

The work of Shirreff \etal\ \cite{shirreff_transmission_2011}, and particularly the predicted transient peak in HIV virulence midway through the epidemic,
highlights the importance of interactions between epidemiological and
evolutionary factors \cite{day_virulence_2004,alizon_price_2009}.
However, despite these studies' attention to detail at the individual
or physiological level, the epidemiological structures used in these
models are relatively simple.

As we discuss in detail below,
existing models of HIV eco-evolutionary dynamics either use implicit
models that incorporate the average effects of within-couple sexual
contact --- without representing the explicit dynamics of pair
formation and dissolution or accounting for extra-partnership contact
--- or use an agent-based formulation with parameters that effectively
lead to random mixing among infected and uninfected individuals. Here
we explore the effects of incorporating \emph{explicit}
epidemiological structure in eco-evolutionary models.

We add complexity to the epidemiological model following the general
approach of Champredon \etal\ \cite{champredon_hiv_2013},
which is in turn based on work of Dietz and Hadeler
\cite{dietz_epidemiological_1988}; individuals
join and leave partnerships at a specified rate, and can have sexual
contact both within and outside of established partnerships. In order
to explore how virulence evolution depends on epidemiological
structure, we consider a series of models with increasing levels of
complexity. In order to avoid dependence of the results on a
particular set of parameters --- as we explain below, finding matching
sets of parameters across models with widely differing epidemiological
structures is challenging --- we evaluate our models across a wide
range of parameters, again following Champredon
\etal\ \cite{champredon_hiv_2013} in using a Latin hypercube
design. For each model run, we compute a set of metrics (minimum
progression time/peak virulence, timing of maximum virulence, equilibrium virulence) that summarize the evolutionary trajectory of a simulated HIV epidemic.

As our primary goal is to explore how different epidemiological
structures (i.e. partnership dynamics and contact structures) affect
our conclusions about the evolution of virulence, our models use a
simplified description of within-host dynamics and heritability
derived from Shirreff \etal's multi-strain evolutionary model
\cite{shirreff_transmission_2011}. Like Shirreff \etal, we use a
simple susceptible-infected-susceptible demographic formulation;
rather than modeling birth and death (or more specifically,
recruitment into the sexually active population and death), we assume
that whenever an individual dies from infection, another enters the
susceptible compartment.

\section*{Materials and Methods}

\subsection*{Infection dynamics}

Like Shirreff \etal\ \cite{shirreff_transmission_2011}, we focus on the evolution of mean $\log_{10}$ set-point viral load, SPVL (which we denote as $\alpha$), rather than the rate of progression to AIDS itself
(hereafter ``virulence'' will refer either to the SPVL, or to 
the rate of progression to AIDS; these two quantities are
deterministically linked in the model) \todo{Doesn't this somewhat contradict with what we say later on? Here, we say we focus on SPVL but later we say we summarize the models using progression time rather than SPVL...}.
In contrast to Shirreff \etal, we use a single-stage disease model instead of accounting explicitly for progression through the three main stages of HIV infection (primary, asymptomatic, and disease), and we use a simple exponentially distributed infectious period instead of a more realistic Weibull-distributed infectious period; we show below that our results are not overly
sensitive to this simplification. We account for varying transmission rates and durations of each disease stage by summing the durations of three stages (again based on Shirreff \etal's model) and taking the duration-weighted average of transmission rates of three stages. Thus the within-couple transmission rate, $\beta$, for our models is given by:
\begin{equation}
\beta (\alpha) = \frac{D_P \beta_P + D_A (\alpha) \beta_A (\alpha) + D_D \beta_D}{D_P + D_A (\alpha) + D_D},
\end{equation}
where the duration of infection ($D_P$ and $D_D$) and rate of transmission ($\beta_P$ and $\beta_D$) of the Primary and Disease stages
of infection are independent of the host's SPVL. Following Shirreff \etal, the duration of infection ($D_A$) and rate of transmission ($\beta_A$) for the Asymptomatic stage are Hill functions of the SPVL:

\begin{equation}
\begin{split}
D_A(\alpha) &= \frac{\tsub{D}{max} D_{50} ^{D_k}}{V_\alpha ^{D_k} + D_{50}^{D_k}}, \\
\beta_A(\alpha) &= \frac{\tsub{\beta}{max} V_\alpha ^ {\beta_k}}%
{V_\alpha^{\beta_k} + \beta_{50} ^{\beta_k}},
\end{split}
\label{eq:hillfuns}
\end{equation}
where $V_{\alpha} = 10^\alpha$. 

The uncoupled and extra-couple transmission rates (i.e., the rates of
transmission among people outside of a stable partnership, or between
people inside of a stable partnership and people other than their
partner) are scaled by
multiplying the within-couple transmission rate $\beta$ by the contact
ratios $c_u/c_w$ and $c_e/c_w$ (see Appendix). Simplifying the model
of HIV pathogenesis from three stages to a single stage could affect
our conclusions about the evolution of virulence (e.g. Kretzschmar and
Dietz \cite{kretzschmar_effect_1998} show that pair formation dynamics
and multiple stages of infectivity have interactive effects on
$\rzero$). However, our simplified model produces results that are
qualitatively similar to those of Shirreff \etal's
\cite{shirreff_transmission_2011} model; when our model is calibrated
to have a similar initial epidemic growth rate $r$, the peak
\Lspvl\ occurs at the same time ($\approx$ 200 years) but slightly
higher (4.6 \Lspvl\ vs. 4.3 \Lspvl, or 7\% higher: \figurename~\ref{fig:panel3}).

\begin{figure}[!ht]
\includegraphics[width=\textwidth]{../figures/fig1.pdf}
\caption{{\bf Baseline dynamics.}
Time series of mean population \Lspvl. (a) Contrast between the three-stage Shirreff model and the single-stage model calibrated to varying initial exponential growth rates, $r$. (b) Effects of varying initial infectious density $I(0)$. (c) Effects of varying initial mean virulence $\alpha(0)$. The $r=0.042$ (orange, dotted) curve in panel (a), calibrated to match the epidemic dynamics of Shirreff \etal's model \cite{shirreff_transmission_2011}, shows that our simplified model can produce similar virulence trajectories. Panels b and c illustrate the sensitivity of virulence trajectories to initial conditions $I(0)$ and $\alpha(0)$, which we hold constant in our simulations.}
\label{fig:panel3}
\end{figure}

\subsection*{Mutation}

Like Shirreff \etal\ \cite{shirreff_transmission_2011} we incorporate a between-host mutation process in the SPVL. We simplify Shirreff \etal's evolutionary model by using a one-to-one genotype-phenotype mapping rather
than allowing for variation in phenotypes of a single genotype.
The mutational process in our model is directly taken from Shirreff \etal. Over the course of infection, mutation occurs within the host. However, it is assumed that SPVL of the strain transmitted by an infected individual is determined by the SPVL at the time of infection and is not further affected by within-host mutation. Instead, the mutational effect takes place in a single
step at the time of transmission. First, the distribution of \Lspvl\ is discretized into a vector:
\begin{equation}
\alpha_i = \tsub{\alpha}{min} + (\tsub{\alpha}{max} - \tsub{\alpha}{min})\frac{i-1}{n-1} \qquad i = 1,2,3, \dots n.
\end{equation}
We have experimented with varying degrees of discretization in the strain distribution (i.e., values of $n$); in our model runs comparing results with Shirreff \etal\ \cite{shirreff_transmission_2011} (\figurename~\ref{fig:panel3}) we use $n=51$ (i.e. a bin width of 0.1 \Lspvl\ for $\alpha$), but reducing $n$ to 21 (bin width = 0.25 \Lspvl) makes little difference; we use this coarser grid for all other simulations reported.

We construct an $n$ by $n$ mutational matrix, $M$ --- which is multiplied with the transmission term ---  so that $M_{ij}$ is the probability that a newly infected individual will have \Lspvl\ of $\alpha_j$ given that the infector has \Lspvl\ of $\alpha_i$. Finally, the probabilities are normalized so that each row sums to 1:
\begin{equation}
M_{ij} = \frac{\Phi(\alpha_j + d/2;i) - \Phi(\alpha_j - d/2;i)}{\Phi(\tsub{\alpha}{max} + d/2;i) - \Phi(\tsub{\alpha}{min} - d/2;i)},
\end{equation}
where $\Phi(x;i)$ is the Gaussian cumulative distribution function with mean $\alpha_i$ and variance of $\sigma_M^2$, and $d = (\tsub{\alpha}{max} - \tsub{\alpha}{min})/(n-1)$. Transmission rate and disease induced mortality rates are discretized as well:
\begin{equation}
\begin{aligned}
\beta_i &= \beta(\alpha_i),\\
\lambda_i &= \frac{1}{D_P + D_A (\alpha_i) + D_D}.
\end{aligned}
\end{equation}

\subsection*{Contact structure and partnership dynamics}

We developed seven multi-strain evolutionary models covering a gamut including Champredon \etal's relatively realistic \cite{champredon_hiv_2013} and Shirreff \etal's relatively simple \cite{shirreff_transmission_2011} epidemiological structures, each of which is based on different assumptions regarding contact structure and partnership dynamics. Specifically, we focus on the effects of the assumptions of (1) instantaneous vs. non-instantaneous partnership formation; (2) zero vs. positive extra-partnership sexual contact and transmission; and (3) homogeneous vs. heterogeneous levels of sexual activity on the evolution of mean \Lspvl.

Our first four models consider explicit partnership dynamics and are based on Champredon \etal's model \cite{champredon_hiv_2013}. The first two (``pair-formation'' or ``pairform'' for short) assume non-instantaneous partnership formation (i.e. individuals spend some time uncoupled, outside of partnerships) and consist of five states that are classified by infection status and partnership status. $S$ is the number of single (uncoupled) susceptible individuals, and $I$ is the number of single infected individuals. $SS$ is the number of concordant negative (susceptible-susceptible) couples, $SI$ is the number of serodiscordant (susceptible-infected) couples, and $II$ is the number of concordant positive (infected-infected) couples. The first (``pairform+epc'') includes extra-partnership contact (with both uncoupled individuals and individuals in other partnerships) whereas the second (``pairform'') only considers within-couple transmission. 

The next two models, which are intended to bridge the gap between models with fully explicit pair-formation dynamics and the simpler, implicit models used by Shirreff \emph{et al.} \cite{shirreff_transmission_2011}, assume instantaneous partnership formation (``instswitch''). The compartmental structure thus omits the single states $S$ and $I$, comprising only the three partnered states: $SS$, $SI$, and $II$. Like the first two models, this pair of models differs in their inclusion of extra-pair contact: the third model (``instswitch+epc'') includes extra-partnership contact (now only with individuals in other partnerships, since uncoupled individuals do not exist in this model) while the fourth (``instswitch'') only considers within-couple transmission. \todo{Although these models can also be achieved by setting the partnership formation rate of the explicit partnership models to a high value (and we have tested that both methods in fact produce same results), we model instantaneous partnership formation models independently in order to avoid scaling of partnership formation rate during model calibration affecting the virulence trajectory.}

The fifth and sixth models represent extreme simplifications of sexual partnership dynamics.  One (``implicit'') is an implicit serial monogamy model based on the epidemiological model used by Shirreff \etal\ \cite{shirreff_transmission_2011}. It is actually a random mixing model that explicitly tracks only the total number of susceptible and infected individuals. However, to reflect the effect of partnership structure, it uses an adjusted transmission rate derived from an approximation of the basic reproduction number of a serial monogamy model with instantaneous pair formation \cite{hollingsworth_hiv1_2008}. The second model of this pair (``random'') is a simple random-mixing model.

Lastly, we add a model of heterogeneity in sexual activity to the pairform+epc model (``hetero''). Individuals are divided into different risk groups based on the sexual activity level; we scale all aspects of sexual activity, assuming that sexual activity level in both within- and extra-couple contacts is directly proportional to number of non-cohabiting (extra-couple and uncoupled) partners per year \cite{omori2015dynamics} (see Appendix). We assume random activity-weighted mixing between risk groups \cite{may_transmission_1988}. While this model lacks some 
important elements, such as age-structured mixing patterns, needed for realistic models of HIV transmission in sub-Saharan Africa, it represents a first step toward assessing the effects of epidemiological complexity. As even the models shown here push the limits of compartmental-based models (the heterogeneity model comprises 24530 coupled ordinary differential equations), adding further complexity will probably require a shift to an agent-based model framework, as well as considerable effort in model calibration \cite{herbeck_hiv_2014,delva_connecting_2016}.

The pairform+epc and heterogeneous models use the basic epidemiological framework of Champredon \etal \cite{champredon_hiv_2013}. Individuals in single compartment acquire a partner at a rate $\rho$, and partnerships dissolve at a rate $c$. Infected individuals in a discordant partnership infect their susceptible partner at a rate $\beta$ (within-couple transmission rate) and susceptible individuals outside the partnership at a rate $c_e$ (extra-couple transmission rate). Likewise, a single infected individual can infect any susceptible individuals at a rate $c_u$ through uncoupled mixing. Extra-couple and uncoupled transmission are modeled in the same way as in Champredon \etal's model. All the details have been adapted to a multi-strain scenario, so that we track (for example) a matrix 
$II_{ij}$ that records the number of concordant, HIV-positive couples in which the two partners have \Lspvl\ of $\alpha_i$ and $\alpha_j$. 
The second through fourth models (pairform, instswitch+epc, instswitch) are derived from the base model by simplifying epidemiological processes (partnership formation and uncoupled/extra-couple contact: see Appendix).

\subsection*{Latin hypercube sampling}

Despite considerable effort \cite{hollingsworth_hiv1_2008,champredon_hiv_2013}, the parameters determining the rate and structure of sexual partnership change and contact are still very uncertain; this led Champredon \etal\ \cite{champredon_hiv_2013} to adopt a Latin hypercube sampling (LHS) strategy \cite{blower_drugs_1991} that evaluates model outcomes over a range of parameter values. In order to make sure that our comparisons among models apply across the entire space of reasonable parameter values, and in order to evaluate the differential sensitivity of different model structures to parameter values, we follow a similar protocol and perform LHS over a parameter set including both the early- and late-stage transmission and duration parameters ($\beta_P$, $D_P$, $\beta_D$, $D_D$) and contact/partnership parameters ($\rho$, $c$, $c_u/c_w$, and $c_e/c_w$). \todo{does this sentence regarding heterogeneity feel redundant?} For the heterogeneity model, the mean and squared coefficient of variation (CV) for the number of non-cohabiting partners are sampled as well. We do not allow for uncertainties in parameters that are directly related to the evolutionary process ($\tsub{\beta}{max}$, $\beta_{50}$, $\beta_k$, $\tsub{D}{max}$, $D_{50}$, $D_k$, $\sigma_M$), instead using Shirreff \etal's point estimates throughout \cite{shirreff_transmission_2011}.

Latin hypercube sampling is done as in Champredon \etal\ \cite{champredon_hiv_2013}. For each parameter, $z$, its range is divided into $N = 1000$ equal intervals on a log scale:
\begin{equation}
z_i = \exp\left(\log(\tsub{z}{min}) + [\log(\tsub{z}{max}) - \log(\tsub{z}{min})] \frac{i-1}{N-1}\right) \qquad i = 1, 2, 3, \dots,N.
\end{equation}
Random permutations of these vectors form columns in a sample parameter matrix; each row contains a different parameter set that is used for one simulation run.

Table~\ref{table:parmsTable} gives the ranges of the model parameters used for LHS. Parameter ranges regarding contact and partnership dynamics ($\rho$, $c$, and $c_e/c_w$) are taken from Champredon \etal\ \cite{champredon_hiv_2013}, whereas those regarding infection ($\beta_P$, $D_P$, $\beta_D$, and $D_D$) are taken from Hollingsworth \etal\ \cite{hollingsworth_hiv1_2008}. The remaining parameters are taken from Shirreff \etal\ \cite{shirreff_transmission_2011}.

%Tables need to be placed after the first paragraph in which they are cited.

\begin{table}[h!]
\caption{Parameter ranges/values.  Values of $c$ and $\rho$ are doubled from those given by Champredon \etal\ because we keep track of individuals in the model, while they keep track of couples. Starred (*) parameters (used in \figurename~\ref{fig:panel3}), and descriptions of Hill function coefficients, are taken from \cite{shirreff_transmission_2011}.}
\centering
\begin{tabular}{c p{2in} c l}
\hline 
Notation & Description & Range/Value & Source\\
\hline % inserts single horizontal line
$\rho$ & Partnership formation rate & 1/10-2/5 per year & \cite{champredon_hiv_2013} \\
$c$ & Partnership dissolution rate & 1/15-1/5 (1.25*) per year & \cite{champredon_hiv_2013} \\
$c_u/c_w$ & Relative contact rate for uncoupled transmission & 1/5-5 & Assumption \\
$c_e/c_w$ & Relative contact rate extra-couple & 0.01-1 & \cite{champredon_hiv_2013} \\
$\beta_P$ & Rate of transmission during primary infection & 1.31-5.09 (2.76*) per year & \cite{hollingsworth_hiv1_2008} \\
$\beta_D$ & Rate of transmission during high transmission disease stage & 0.413-1.28 (0.76*) per year & \cite{hollingsworth_hiv1_2008} \\
$D_P$ & Duration of primary infection & 1.23/12-6/12 (0.25*) years & \cite{hollingsworth_hiv1_2008} \\
$D_D$ & Duration of high transmission disease stage & 4.81/12-14/12 (0.75*) years & \cite{hollingsworth_hiv1_2008} \\
$\tsub{\beta}{max}$ & Maximum rate of transmission during asymptomatic stage & 0.317 per year & \cite{shirreff_transmission_2011} \\
$\beta_{50}$ & SPVL at which infectiousness is half maximum & 13938 copies per ml & \cite{shirreff_transmission_2011} \\
$\beta_k$ & Hill coefficient: steepness of increase in infectiousness as a function of SPVL & 1.02 & \cite{shirreff_transmission_2011} \\
$\tsub{D}{max}$ & Duration of primary infection & 25.4 years & \cite{shirreff_transmission_2011} \\
$D_{50}$ & SPVL at which duration of asymptomatic infection is half maximum & 3058 copies per ml & \cite{shirreff_transmission_2011} \\
$D_{k}$ & Hill coefficient: steepness of decrease in duration as a function of SPVL & 0.41 & \cite{shirreff_transmission_2011} \\
$\sigma_M$ & Mutation standard deviation of $\log_{10}$ SPVL & 0.12 & \cite{shirreff_transmission_2011} \\
$\tsub{\alpha}{min}$ & Minimum $\log_{10}$ SPVL & 2 & \cite{shirreff_transmission_2011}\\
$\tsub{\alpha}{max}$ & Maximum $\log_{10}$ SPVL & 7 & \cite{shirreff_transmission_2011}\\
$n$ & Number of strains & 21 (51*) & Assumption\\
$\mu$ & Mean number of non-cohabiting sexual partners & 0.103 - 1.206 & \cite{omori2015dynamics}\\
$\kappa$ & Squared coefficient of variation of number of non-cohabiting sexual partners & 0.01 - 100 & Assumption\\
\hline
\end{tabular}
\label{table:parmsTable}
\end{table}

One parameter in our model, the ratio of uncoupled to within-couple transmission $c_u/c_w$, is needed to more flexibly contrast uncoupled and extra-couple transmission dynamics within multi-strain models (Appendix S1);
it appears neither in either Shirreff \etal\ nor Champredon \etal's models,  so we need to pick a reasonable range for it. Champredon \etal\ \cite{champredon_hiv_2013} assume that the effective within-couple contact rate and effective uncoupled contact rate have the same range of 0.05 - 0.25.  Given Champredon \etal's parameter range, the possible maximum and minimum values of $c_u/c_w$ are 5 and 1/5. Therefore, we use 1/5-5 as the range for the parameter $c_u/c_w$. Although this adds more uncertainty to the parameter $c_u$ --- Champredon \etal's range implies a 5-fold difference whereas ours gives a 25-fold difference --- we consider the wider range appropriate, as little is not much known about the uncoupled transmission rate.

Two parameters, mean and the squared coefficient of variation (CV) of number of non-cohabiting partners, are sampled for heterogeneity in sexual activity.
%%  BMB: I don't think we need to go into detail about variance vs CV
To allow for a wide range of uncertainty, range for the mean number of non-cohabiting partners was taken from unmarried men, as that was the group with the largest variability \cite{omori2015dynamics}. 
Omori \etal \cite{omori2015dynamics} give a very wide
range for the coefficient of variation ($\approx$ 0 - 20, corresponding
to squared CV range of 0-400):
we narrowed this range for $\textrm{CV}^2$ to 0.01-100.
At the bottom end of the range, estimating that a group behaves
perfectly homogeneously ($CV=0$) is likely to be a sampling artifact;
at the upper end, the estimate is also likely to be noisy because
of the low mean value among married females (who have the largest
range of CV). 
We assume that the number of non-cohabiting partners follows a Gamma distribution.

\subsection*{Simulation runs}

One of the most difficult parts of model comparison is finding
parameter sets that are commensurate with many different model
structures. For the most part, our models are too complex to easily
derive analytical correspondences among them. Given a numerical
criterion, such as $r$ (initial exponential growth rate) or $\rzero$ 
(intrinsic reproductive number), we can adjust one or more
parameters by brute force to ensure that all of the models match
according to that criterion. While $\rzero$ is often considered
the most fundamental property of an epidemic, and might thus seem to
be a natural matching criterion, here we focus on matching the initial
growth rate $r$ for several reasons. First, our primary interest is in
the transient evolutionary dynamics of virulence, which are more
strongly affected by $r$ than $\rzero$. Second, $r$ is 
more directly observable in real epidemics; $r$ can be estimated by
fitting an exponential curve to the initial incidence or
prevalence curves \cite{ma_estimating_2014}, while $\rzero$
typically requires either (1) knowledge of \emph{all} epidemic
parameters or (2) calculations based on
$r$ and knowledge of the serial interval or generation interval of the
disease \cite{wallinga_how_2007}. Thus, we scale parameters so that
every run has the same initial exponential growth rate in 
disease incidence.

In order to allow for all models to have equal initial exponential
growth rate, $r$, we need to pick a parameter, $s$, such that
$\lim_{s\to 0} r(s) = 0$ and $\lim_{s\to\infty} r(s) = \infty$. As
adjusting either partnership change rate (i.e. partnership formation
and dissolution rate) or transmission rate fails this requirement for
some of our models, we scaled partnership change rate and
dissolution rate by the same factor of $\gamma$: $\tsub{\beta}{adj} =
\gamma \tsub{\beta}{base}$, $\tsub{c}{adj} = \gamma \tsub{c}{base}$,
$\tsub{\rho}{adj} = \gamma \tsub{\rho}{base}$. Since transmission rate
is adjusted by the scale of $\gamma$, uncoupled and extra-couple
transmission rates are adjusted as well. For the instantaneous-switching
and implicit models, none of which track single individuals, 
only the transmission rate and partnership
dissolution rate (in this case equivalent to the partnership change
rate) are adjusted.

We run each model for each of 1000 parameter sets chosen by Latin hypercube sampling, with fixed starting conditions
of mean \Lspvl\ of 3 and epidemic size of $10^{-4}$. After each run, initial exponential growth rate is calculated. Then, parameters are scaled so that the initial exponential growth rate is scaled to 0.04, a value that approximates the growth rates of Shirreff \etal's original models.
For calibration purposes, we run each model for only 500 years
(full simulations are run for 4000 years), which is always long
enough to capture the exponential growth phase of the model. 
We use a 4/5 order 
Runge-Kutta method (\texttt{ode45} from the \texttt{deSolve} package
\cite{soetaert_solving_2010}) for all simulations. \todo{mention that we got rid of error runs for hetero model?}

Although each disease strain's core characteristic is its SPVL, the
SPVL has one-to-one correspondences (based on eq.~\ref{eq:hillfuns})
with both the expected time to progression to AIDS and with the rate
(probability per unit time) of HIV transmission. Because the time to
progression (measured in years) is easier to interpret than
SPVL (measured in \Lspvl\ units), we summarize the virulence
trajectories for each model run in terms of time to progression
rather than SPVL. Because the  time to progression is inversely
related to SPVL (increasing SPVL decreases the time to progression),
the time to progression is technically measuring inverse
virulence rather than virulence (we did not think that 
reporting virulence as the rate of progression to AIDS, in units
of $\textrm{years}^{-1}$, would help interpretability).
For each model we derive the following summary statistics:
minimum expected time to progression;
time at which this minimum occurs
(corresponding to peak virulence --- this is also the time at which the
maximum rate of progression, maximum SPVL, and maximum transmission rate
occur); equilibrium time to progression; 
and the ratio of progression time at its minimum to the equilibrium
value. Equilibrium progression time is calculated after 4000 years of simulated
time. Although most simulations reach equilibrium much earlier, we set our time horizon at a much later date as some simulation runs have slow rate of evolution depending on the parameter set and model assumptions.

Knowing the minimum progression time, timing of the minimum progression time/peak virulence, and equilibrium progression time provide sufficient detail to identify the overall shape of the virulence trajectory.
In particular, knowing the timing of the peak virulence (how many years
into the epidemic the virulence peaks) can help epidemiologists
guess whether the virulence of an emerging pathogen is likely (1)
to have peaked early, possibly even before the pathogen is detected
spreading in the population, and decline over the remaining course
of the epidemic; (2) to increase, peak, and decline over the
foreseeable future; or (3) to increase very slowly, peaking only
in the far future. To the extent that our simplistic model for HIV
reflects reality, we would take the peak time of 150-300 years 
(\figurename~\ref{fig:panel3}c) to mean that, in the absence of
treatment, the epidemic would probably still be increasing in virulence.

% Results and Discussion can be combined.
\section*{Results}

Our simplifications of Shirreff \etal's model \cite{shirreff_transmission_2011} reproduce its qualitative behaviour --- in particular, its predictions of virulence dynamics --- reasonably well. As $r$ decreases from 0.084 to 0.42 (the latter value matching the initial rate of increase in prevalence in Shirreff \etal's full model) the initial trajectory of increasing virulence brackets the rate from the original model (\figurename~\ref{fig:panel3}a). However, our model produces lower peak virulence ($\approx 4.3$ vs. $\approx 4.6$ \Lspvl) 
and equilibrium virulence ($\approx 4.25$ vs. $\approx 4.5$ \Lspvl) than Shirreff's, even for matching initial incidence trajectories (i.e., $r=0.042~\textrm{year}^{-1}$).

Changing the initial infectious density ($I(0)$), while it produces the expected changes in the initial epidemic trajectory (Supplementary material), has little effect on the virulence trajectory, making the virulence peaks slightly later and larger as $I(0)$ decreases. Decreasing $I(0)$ allows a longer epidemic phase before the transition to endemic dynamics (\figurename~\ref{fig:panel3}b). Decreasing the initial virulence
also leads to progressively later, larger peaks in virulence (\figurename~\ref{fig:panel3}c).

Across the entire range of parameters covered by the LHS analysis, all of the classes of models we considered produce qualitatively similar virulence trajectories, which we quantify in terms of the expected time of progression
to AIDS (\figurename~\ref{fig:virtraj}: lower progression time corresponds
to higher virulence). Although the speed of virulence evolution varies, leading to wide variation in the minimum expected progression time (means ranging from approximately 6 to 12 years), virulence peaks in all models between 200 and 300 years.

\begin{figure}[!ht]
\includegraphics[width=\textwidth]{../figures/fig2.pdf}
\caption{{\bf Envelopes of virulence trajectories (expected
time of progression to AIDS) under all models.}
All models were run until $t=4000~\textrm{years}$; truncated series are shown here.}
\label{fig:virtraj}
\end{figure}

Our chosen summary statistics (peak time, minimum expected progression time, equilibrium expected
progression time, and relative progression time) all vary considerably across models
(\figurename~\ref{fig:unidist}).
We first consider the models of intermediate realism: implicit,
instantaneous-switching with and without extra-pair contact, and
pair formation without extra-pair contact. Some parameter
sets for these models lead to low equilibrium virulence ($\approx 18$ years
to progression);
these same sets lead to correspondingly low
peak virulence (16 years to progression) and early peak times (before 200 years: \figurename~\ref{fig:pairplot}).
At the opposite extreme, parameter sets that produce high equilibrium virulence (8 years to progression)
also produce late peaks ($> 200~\text{years}$) and
high peak virulence (4 years to progression).
The pair-formation without extra-pair contact and implicit models
occasionally have parameter sets that select for such low virulence across
the board that they never exceed their initial virulence, leading to a tail
of peak times near zero.

\begin{figure}[!ht]
\includegraphics[width=\textwidth]{../figures/fig3.pdf}
\caption{{\bf Univariate distributions of summary statistics.}
The distribution of equilibrium expected progression
time (lower left panel) for the random mixing model is very narrow, and has been replaced by a point in order to preserve the vertical axis scaling.}
\label{fig:unidist}
\end{figure}

\begin{figure}[!ht]
\includegraphics[width=\textwidth]{../figures/fig4.pdf}
\caption{{\bf Pairs plot: bivariate relationships among summary statistics for each model structure.}
Dashed line in equilibrium vs. peak virulence plot shows 1:1 line. 100 values were sampled from each model to allow for clearer distinction between the models}
\label{fig:pairplot}
\end{figure}

The most striking aspect of the univariate comparisons in
\figurename~\ref{fig:unidist}, (and the bivariate comparisons in
\figurename~\ref{fig:pairplot}) is the similarity between the results of the
least (random mixing) and the most complex (pair formation with
extra-pair contact, pairform+epc with heterogeneity) models. The random-mixing model has the lowest variability, because it is unaffected by uncertainty in pair formation and extra-pair contact parameters, but otherwise the virulence
dynamics of these three extreme models are remarkably similar.
This phenomenon is driven by the strong effects of extra-pair contact in the
model with explicit pair formation and extra-pair contact 
(``pairform+epc'' in \figurename{}s~\ref{fig:virtraj}-\ref{fig:plot_sens}). When individuals spend time uncoupled between
partnerships, and when these single individuals can transmit disease
to coupled individuals, the resulting unstructured mixing overwhelms
the effect of structured mixing within couples, leading to mixing
that is effectively close to random.
Once unstructured mixing is strong, adding realistic heterogeneity
of mixing to the model has little effect other than increasing
the variability in the outcomes.

These differences are practically as well as 
scientifically important. The random-mixing, pairform+epc,
and heterogeneous models all predict rapid
progression to AIDS at the virulence peak
(median/95\% CI = 6.1 (5.7-6.3), 
6.02 (5.04-7.7), 6.03  (4.8-9.2)). 
In contrast, 
the implicit model predicts minimum progression times about twice as long:
12.5 (9.6-15.6) years. The corresponding differences in 
within-couple transmission
probability are even more extreme, about a fourfold difference:
0.249 (0.24-0.26), 0.252 (0.19-0.28), and 0.252 (0.15-0.28) per year for the 
random and pairform+epc models vs. 0.059 (0.02-0.13) per year
for the implicit model (see Appendix for 
plots showing univariate summaries
of \Lspvl\ and transmission probability).

\begin{figure}[!ht]
\includegraphics[width=\textwidth]{../figures/fig5.pdf}
\caption{{\bf Sensitivity plot.}
For most parameters in the Latin hypercube sample and each summary statistic, the figure shows the distribution (points) and trend (smooth line) of the summary statistic as a function of the \emph{unscaled} parameter value, i.e. prior to adjusting the parameters to achieve the standard initial epidemic growth rate.}
\label{fig:plot_sens}
\end{figure}

The bivariate relationships (\figurename~\ref{fig:pairplot}) help distinguish the results of 
different models with similar univariate dynamical summaries. While the
relationship between equilibrium progression time and peak time is
similar for all model structures (top left panel), the other
relationships show more variation. In particular, the implicit
and pair-formation (without extra-pair contact) are very similar
to each other, but distinct from the other models. We still do
not have a convincing explanation for this distinction; we
would have expected the implicit model to be most similar to the
the instantaneous-switching model without extra-pair contact,
which most closely matches its derivation. However, we note
that the implicit model derivation is based on defining
the force of infection to match a scaled version of $\rzero$,
and as such would be expected to match the equilibrium behaviour
but not necessarily the epidemic-phase behaviour of a model
with explicit partnership dynamics.

Finally, the sensitivity plot (\figurename~\ref{fig:plot_sens}) shows the effects 
of each parameter on the summary statistics. In almost every case the
effects of the parameters are monotonic; note that
the plot shows the effects of the \emph{unscaled} parameters, i.e.
before they have been calibrated to achieve a standard initial epidemic
growth rate.
Increases in the transmission rates ($\beta_P$, $\beta_D$)
and durations ($D_P$, $D_D$) in the primary and disease stages generally
decrease the equilibrium virulence, peak virulence, and peak time,
although the random and pair-formation+epc 
models have high, relatively
constant values with respect to these parameters
(because the patterns are so similar across this set of parameters,
\figurename~\ref{fig:plot_sens} shows only $D_P$).

The partnership dissolution rate ($c$), which essentially
acts as a contact rate in the model,
increases virulence and peak time in almost all
cases, although the pair-formation+epc
model is again relatively insensitive.
The ratio of extra-pair to within-pair contact ($c_e/c_w$) affects
virulence in the instantaneous-switching+epc model, but not the pair-formation+epc
model (probably because the uncoupled individuals present in the pair-formation+epc
model make extra-pair contact by coupled individuals less important).
Surprisingly, once calibration
is taken into account, the remaining parameters have little effect overall.
The rate of partnership formation ($\rho$) 
has little impact on the models with finite pair-formation times.
The relative rate of uncoupled contact ($c_u/c_w$) slightly decreases the
minimum and equilibrium progression time and delays the peak time in the
pair-formation+epc model, but neither the uncoupled contact rate nor
the mean ($\mu$) or CV$^2$ of the number of non-cohabiting sexual
partners has much systematic effect in the heterogeneous model.

\section*{Discussion}

All models must simplify the world.  Many constraints --- among them data
availability, computation time, and code complexity --- drive the need
for parsimony, with different constraints applying in different
contexts. The critical question that modelers must ask is whether the
simplified model gives adequate answers, or whether the
simplifications lead to qualitative or quantitative errors.
This question is especially important for modelers who
are hoping that their conclusions will guide management decisions.

In the particular example of HIV virulence eco-evolutionary dynamics
and the complexity of epidemiological structures
we reach the slightly ironic conclusion that the
effort put into building a more realistic model essentially cancels
out, putting us back where we started when used a naive random-mixing
contact model.
However, we are not quite back where we started, as the
complex models lead to wider, presumably more realistic
confidence intervals on the predictions.
In general, unstructured mixing --- whether occurring through 
purely random mixing, or through extra-pair contact and contact
among people outside of stable partnerships --- tends to drive
faster virulence evolution, leading to higher peak virulence and 
lower times to progression at the peak time.

%In Herbeck \etal's \cite{herbeck_hiv_2014} network model of partnerships, the partnership duration is set to 1 day --- very unrealistic in epidemic terms, but perhaps
%actually more true to real-world HIV epidemiological dynamics than a
%model with realistic partnership durations that neglects extra-pair
%contact \cite{herbeck2016evolution}. 
Taking further steps to make the model even more realistic
might add further structure,
making the random-mixing model predictions less accurate. For
example, our model forms partnerships randomly, and assumes that
extra-pair contact is randomly mixing across the population;
one could instead model extra-pair contact as arising from
multiple concurrent partnerships (some, such as contact with sex
workers, of very short duration) and/or more structured partnership
formation (by age, ethnicity, or behaviour group). The effects of
other realistic complications such as explicit modeling of two
sexes (both in contact structure and differential transmission
probabilities), temporal and spatial variation in epidemic processes,
or presence of genetic variation in hosts are harder to predict.

Parameterization is one of the biggest challenges of epidemiological
modeling. In addition to following Champredon \etal\ \cite{champredon_hiv_2013} 
by doing Latin hypercube
sampling across a wide range of epidemiological parameters, we 
calibrated each set of parameters to the same initial epidemic
growth rate, chosen to match the results of previous models
\cite{shirreff_transmission_2011}.  Previous models 
in this area have drawn their
parameters from cohort studies from the 1990s
\cite{wawer2005rates,hollingsworth_hiv1_2008}
rather than doing any explicit calibration to epidemic curves,
but they give reasonable order-of-magnitude
growth rates ($\approx 0.04~\textrm{year}^{-1}$)
for the early stages of the HIV epidemic (although considerably
lower than estimates of $\approx 0.07-0.1~\textrm{year}^{-1}$
based on population genetic reconstructions \cite{faria_early_2014}).
However, our reason for calibrating was not to match any
specific observed epidemic, but rather to make sure that
we were making meaningful comparisons across a range of
models with radically different epidemiological structures, and
hence involving different interpretations of the same quantitative
parameters.  For example, in models with instantaneous switching the
partnership dissolution rate $c$ is identical to the partnership
formation rate; in models with explicit partnership formation,
the partnership formation rate is also $c$ at equilibrium,
but might vary over the course of an epidemic.
It is not obvious whether models with equal parameters but
different structures should be directly compared; calibration
solves this problem.

More generally, any model that wants to be
taken seriously for management and forecasting purposes should
be calibrated to \emph{all} available data, using informative
priors to incorporate both realistic distributions of uncertainty
for all parameters from independent measurements \cite{elderd_uncertainty_2006}
and calibration from population-level observations of epidemic
trajectories. Such a procedure would also be an improvement on the common --- although not universal --- %
practice, which we have followed here,
of assessing uncertainty over uniform ranges rather than
using distributions that allow more continuous variation in support over
the range of a parameter.

Researchers have documented that HIV virulence and set-point viral
load are changing, on time scales comparable to those portrayed here
(e.g., compare \figurename~\ref{fig:virtraj} to Herbeck \etal's
estimated rate of change of 1.3 \Lspvl\ per century [95\% CI -0.1 to
  3] \cite{herbeck_is_2012}), and have begun to build relatively realistic models that
attempt to describe how interventions such as mass antiretroviral
therapy (ART) can be expected to change the trajectory of virulence
evolution \cite{payne_impact_2014,roberts2015impact,herbeck2016evolution}.  While these
efforts are well-intentioned, we caution that epidemiological and
other structural details that are currently omitted from these models
could significantly change their conclusions.

\section*{Acknowledgements}
We would like to thank Christophe Fraser and
David Champredon for access to simulation code;
this work was funded by NSERC Discovery Grant 386590-2010.


\section*{Supporting Information}

% Include only the SI item label in the paragraph heading. Use the \nameref{label} command to cite SI items in the text.
\paragraph*{Appendix S1: model details}
\label{S1_Appendix}

Since we use multi-strain models in which the distribution of \Lspvl\ has been discretized into a vector, we use a matrix notation to describe our models. The five states described in the \emph{Methods} section are replaced with the following notations: $S$, $I_i$, $SS$, $SI_i$, $II_{ij}$, where the subscripts denote the strain with which an individual is infected. For example, $I_i$ is number of infected individuals with \Lspvl\ of $\alpha_i$, and $II_{ij}$ is the number of concordant, HIV-positive couples in which the two partners have \Lspvl\ of $\alpha_i$ and $\alpha_j$ (independent of order; $II_{ij}$ is synonymous with $II_{ji}$). 
Below, we use the Kronecker delta (i.e. $\delta_{ij}=1$ if $i=j$ and 1 otherwise) in a slightly non-standard fashion as an exponent, e.g. $2^{\delta_{ij}}$, to set a value to 2 when $i=j$ and 1 otherwise.

\subsection*{Models 1 (``pairform+epc'') and 2 (``pairform'')}

\subsubsection*{Partnership dynamics}

Single individuals acquire partners at per-person rate $\rho$. Partnership formation rates for $S$ and $I_i$ are $\rho S$ and $\rho I_i$, respectively. We follow Champredon \etal \cite{champredon_hiv_2013} in assuming that single individuals are distributed into coupled states with pair-formation (PF) rates as follows:

\begin{equation}
\begin{aligned}
\PF(SS) &= \frac{\rho S \cdot S}{2 (S + \sum_k I_k)},\\
\PF(SI_i) &= \frac{\rho S \cdot I_i}{S + \sum_k I_k},\\
\PF(II_{ij}) &= \khalf \cdot \frac{\rho I_i \cdot I_j}{S + \sum_k I_k}.
\end{aligned}
\end{equation}

Partnerships dissolve at per-partnership rate $c$: the dissolution rates for $SS$, $SI_i$, and $II_{ij}$ pairs are $c SS$, $c SI_i$, and $c II_{ij}$ respectively. Unlike a single-strain model, where both individuals leaving the $II$ partnership would enter $I$, we have to account for strains with which the individuals in concordant partnership are infected (i.e. both partners in $II_{ii}$ enter $I_i$ whereas one partner in $II_{ij}$ enters the $I_i$ compartment while the other enters $I_j$). Thus, coupled individuals are distributed into single states through partnership dissolution (DS) rates:

\begin{equation}
\begin{aligned}
\DS(S) &= 2 c SS + \sum_k c SI_k, \\
\DS(I_i) &= c SI_i + \sum_k 2^{\delta_{ik}} c II_{ik}.\\
\end{aligned}
\end{equation}

Combining the partnership formation and dissolution processes yields the following equation:

\begin{equation}
\begin{aligned}
S' &= - \rho S + 2 c SS + \sum_k c SI_k \\
I_i' &= - \rho I_i + c SI_i + \sum_k 2^{\delta_{ik}} c II_{ik}\\
SS' &= \frac{\rho S \cdot S}{2 (S + \sum_k I_k)} - c SS\\
SI_i' &= \frac{\rho S \cdot I_i}{S + \sum_k I_k} - c SI_i\\
II_{ij}' &= \khalf \cdot \frac{\rho I_i \cdot I_j}{S + \sum_k I_k} - c II_{ij}
\end{aligned}
\end{equation}

\subsubsection*{Pair-formation models: infection dynamics}

Within-couple transmission (WT) occurs in both models. An infected partner in $SI$ partnership transmits virus to a susceptible partner at per-partnership rate $\beta$: $\WT(SI_i) = - \beta_i SI_i$. Since we assume that mutation occurs, $II_{ij}$ pairs, where $i \neq j$, can be formed from either $SI_i$ or $SI_j$ partnerships: $\WT(II_{ij}) = M_{ij} \beta_i SI_i + M_{ji} \beta_j SI_j$. On the other hand, $II_{ii}$ can only be formed from an $SI_i$ partnership: $\WT(II_{ii}) = M_{ii} \beta_i SI_i$. Using the Kronecker delta notation, we obtain the following set of equations for within-couple transmission dynamics:

\begin{equation}
\begin{aligned}
\WT(SI_i) &= - \beta_i SI_i,\\
\WT(II_{ij}) &=  \khalf \cdot (M_{ij} \beta_i SI_i + M_{ji}. \beta_j SI_j)
\end{aligned}
\end{equation}

Champredon \etal\ \cite{champredon_hiv_2013} define the proportion of infectious extra-couple and uncoupled contact through the following term:

\begin{equation}
P = \frac{c_u I + c_e (SI + 2 II)}{c_u (S + I) + 2 c_e(SS + SI + II)}.
\end{equation}
The effective uncoupled, $c_u$, and extra couple, $c_e$, contact rates are the product of uncoupled/extra-couple contact rate $\times$ rate of transmission per contact. Therefore, the transmission rate per contact term in $c_u$ and $c_e$ is canceled out in the equation above. Using this property, we modify the equation above as follows:

\begin{equation}
P = \frac{r_u I + r_e (SI + 2 II)}{r_u (S + I) + 2 r_e(SS + SI + II)},
\end{equation}
where $r_u = c_u/c_w$ and $r_e = c_e/c_w$ are the relative uncoupled/extra-couple contact rates. This simplification is useful in a multi-strain model since we cannot multiply a vector by a scalar value (e.g. $c_u S$ in denominator) if we use Champredon \etal's equation in its original form. Extending the above equation to the multi-strain model so that $P_i$ represents the proportion of the extra-couple and uncoupled contact of an infected individual with strain $i$, we obtain:

\begin{equation}
P_i = \frac{r_u I_i + r_e (SI_i + \sum_k (II_{ik} + \delta_{ik} II_{ik}))}{r_u (S + \sum_k I_k) + r_e(2 SS + \sum_k 2 SI_k + \sum_l \sum_k 2^\delta_{lk} II_{lk} )}.
\end{equation}
Using the equation above, we can model extra-pair transmission (ET). For convenience, uncoupled and extra-couple transmission rates, $c_u$ and $c_e$, will be replaced with $U_i = r_u \beta_i$ and $E_i = r_e \beta_i$ hereafter.

Single susceptible individuals become infected through uncoupled contact at per-person rate $\sum_k P_k U_k$ and enter the single infected state. Through mutation, newly infected individuals are distributed into single infected compartments with different strains: $\ET(I_i) = \sum_k M_{ki} P_k U_k S$. Either partner in an $SS$ partnership can become infected at per-person rate $\sum_k P_k E_k$, and partnership state changes to an $SI$ partnership at the total rate of $\sum_i 2 P_i E_i SS$. The formation of $SI_i$ partnerships is similar to the process through which single susceptible individuals are distributed into single infected compartments: $\ET(SI_i) = \sum_k 2 M_{ki} P_k E_k SS$. Lastly, the susceptible partner in an $SI$ partnership can become infected from extra-couple contacts at a per-person rate of $\sum_k P_k E_k$, so that the partnership changes to an $II$ partnership. As in the previous cases, $SI_i$ partnerships are lost at a rate of $\sum_k P_k E_k SI_i$. The mutation process is similar to that of within-couple transmission. The only difference is that the \Lspvl\ of a newly infected partner is not determined by its social partner but from an extra-couple partner (i.e. the term $P_i$): $\ET(II_{ij}) = (\frac{1}{2})^{\delta_{ij}}(\sum_k (M_{kj} P_k E_k SI_i + M_{ki} P_k E_k SI_j))$. Combining these equations we get the following set of equations that describe the complete transmission dynamics:

\begin{equation}
\begin{aligned}
S' =& - \sum_k P_k U_k S,\\
I_i' =& \sum_k M_{ki} P_k U_k S,\\
SS' =&  - \sum_i 2 P_i E_i SS, \\
SI_i' =& \sum_k 2 M_{ki} P_k E_k SS - \beta_i SI_i - \sum_k P_k E_k SI_i,\\
II_{ij}' =& \khalf \cdot (M_{ij} \beta_i SI_i + M_{ji}, \beta_j SI_j) + \khalf \cdot (\sum_k (M_{kj} P_k E_k SI_i\\
&+ M_{ki} P_k E_k SI_j)).
\end{aligned}
\end{equation}

\subsubsection*{Pair formation models: Disease induced mortality}

The per-person disease induced mortality (DM) rate, $\lambda$, is given by taking the reciprocal of the total duration of the infection: $\lambda_i = 1/(D_A + D_P(\alpha_i) + D_D)$. Since we assume an SIS formulation, where infected individuals that die from infection are immediately replaced by an individual in the single susceptible compartment, we obtain the following equation for single infected individuals:

\begin{equation}
\begin{aligned}
\DM(S) &= \sum_k \lambda_k I_k, \\
\DM(I_i) &= - \lambda_i I_i. \\
\end{aligned}
\end{equation}

If an infected individual in a partnership dies, the partnership dissolves. Thus, an $SI_i$ partnership dissolves at per-partnership rate $\lambda_i$, and the susceptible partner enters the single susceptible compartment at rate $\lambda_i SI_i$ (due to the SIS formulation, the infected partner that dies also gives rise, at an equal rate, to an individual entering the single susceptible compartment):

\begin{equation}
\begin{aligned}
\DM(S) &= \sum_k 2 \lambda_k SI_k, \\
\DM(SI_i) &= - \lambda_i SI_i.
\end{aligned}
\end{equation}

Similarly, since $II_{ij}$ partnerships are composed of two infected partners, they dissolve at a per-partnership rate $(\lambda_i + \lambda_j)$. However, two cases, when $i \neq j$ and $i = j$, must be considered separately. When an $II_{ij}$ partnership dissolves due to disease-induced mortality, where $i \neq j$, the death of the partner with strain $i$ causes its partner to enter $I_j$ compartment at rate $\lambda_j II_{ij}$, and vice versa. When an $II_ii$ partnership dissolves, the death of either partner causes the other partner to enter the $I_i$ compartment at rate $\lambda_i II_{ii}$, which sums up to $2\lambda_i II_{ii}$. Combining these dynamics yields:

\begin{equation}
\begin{aligned}
\DM(S) &= \sum_l \sum_k  2^{\delta_{lk}} \lambda_k II_{lk}, \\
\DM(I_i) &=  \sum_k 2^{\delta_{ik}} \lambda_k II_{ik}, \\
\DM(II_{ij}) &= -(\lambda_i + \lambda_j) II_{ij}.
\end{aligned}
\end{equation}

Finally, combining all these equations give us the full model, which is Model 1. We can simply drop the uncoupled and extra-couple transmission terms to obtain model 2:

\begin{equation}
\begin{aligned}
S' =& - \rho S + 2 c SS + \sum_k c SI_k - \sum_k P_k U_k S + \sum_k \lambda_k I_k \\
&+ \sum_k 2 \lambda_k SI_k + \sum_l \sum_k  2^{\delta_{lk}} \lambda_k II_{lk}\\
I_i' =&  - \rho I_i + c SI_i + \sum_k 2^{\delta_{ik}} c  II_{ik} + \sum_k M_{ki} P_k U_k S- \lambda_i I_i \\
&+ \sum_k 2^{\delta_{ik}} \lambda_k II_{ik} \\
SS' =& \frac{\rho S \cdot S}{2 (S + \sum_k I_k)} - c SS - \sum_i 2 P_i E_i SS \\
SI_i' =& \frac{\rho S \cdot I_i}{S + \sum_k I_k} - c SI_i - \beta_i SI_i + \sum_k 2 M_{ki} P_k E_k SS - \sum_k P_k E_k SI_i   \\
&- \lambda_i SI_i\\
II_{ij}' =& \khalf \cdot \frac{\rho I_i \cdot I_j}{(S + \sum_k I_k)} - c II_{ij} + \khalf \cdot (M_{ij} \beta_i SI_i + M_{ji} \beta_j SI_j) \\
&+ \khalf \cdot (\sum_k (M_{kj} P_k E_k SI_i + M_{ki} P_k E_k SI_j)) -(\lambda_i + \lambda_j) II_{ij}
\end{aligned}
\end{equation}

\subsection*{Models 3 (``instswitch'') and 4 (``instswitch'')}
\subsubsection*{Partnership dynamics}

Since model 3 and 4 assume instantaneous partnership formation, there are only three states: $SS$, $SI_i$, and $II_{ij}$. Partnership dissolution rates are equal to those of model 1 and 2: $\DS(SS) = -cSS$, $\DS(SI_i) = - cSI_i$, and $\DS(II_{ij}) = - c II_{ij}$. Once individuals leave a partnership, they are instantaneously distributed into coupled states. In order to make the equations simpler, we introduce the following two terms: $X$ and $Y_i$, where $X$ denotes the number of susceptible individuals that leave the partnership at a given time, and $Y_i$ the number of infected individuals with \Lspvl\ of $\alpha_i$ who leave partnership at a given time. These temporarily single individuals then form couples through the same partnership formation rule described in the previous section:

\begin{equation}
\begin{aligned}
X &= 2 c SS + \sum_k c SI_k \\
Y_i &= c SI_i + \sum_k 2^{\delta_{ik}} c II_{ik} \\
SS' &= - c SS + \frac{X^2}{2 (X + \sum_k Y_k)}\\
SI_i' &= - c SI_i + \frac{X Y_i}{X + \sum_k Y_k}\\
II_{ij}' &= - c II_{ij} +\khalf \frac{Y_i Y_j}{X + \sum_k Y_k}.
\end{aligned}
\end{equation}

\subsubsection*{Instantaneous-switching models: Infection dynamics}

Model 3 and 4 share the within-couple transmission term with model 1 and 2. Since there is no single (uncoupled) state, only extra-couple transmission occurs:

\begin{equation}
P_i = \frac{r_e (SI_i + \sum_k (II_{ik} + \delta_{ik} II_{ik}))}{r_e(2 SS + \sum_k 2 SI_k + \sum_l \sum_k (2^\delta_{kl} II_{lk}) )}.
\end{equation}
Movement from $SS$ state to $SI$ state and $SI$ to $SS$ is modeled through the same equation that is used in models 1 and 2.

\subsubsection*{Instantaneous-switching models: Disease induced mortality}

Disease-induced mortality is modeled similarly to model 1 and 2. However, as the single state does not exist in model 3 and 4, individuals that leave partnerships due to death of their partners enter temporary compartments and form partnerships instantly:

\begin{equation}
\begin{aligned}
X &= \sum_k 2 \lambda_k SI_k + \sum_l \sum_k 2^{\delta_{lk}}  \lambda_k II_{lk}, \\
Y_i &=  \sum_k  2^{\delta_{ik}}  \lambda_k II_{ik}, \\
SS' = &= \frac{X^2}{2 (X + \sum_k Y_k)},\\
SI_i' &= - \lambda_i SI_i + \frac{X Y_i}{X + \sum_k Y_k},\\
II_{ij}' &= -(\lambda_i + \lambda_j) II_{ij} + \khalf \cdot \frac{Y_i Y_j}{X + \sum_k Y_k}.
\end{aligned}
\end{equation}

Combining all these dynamics, we have model 3. If we remove extra-couple transmission, we have model 4.

\begin{equation}
\begin{aligned}
X =& 2 c SS + \sum_k c SI_k + \sum_k 2 \lambda_k SI_k + \sum_l \sum_k 2^{\delta_{lk}}  \lambda_k II_{lk},\\
Y_i =& c SI_i + \sum_k 2^{\delta_{ik}}  c II_{ik} + \sum_k  2^{\delta_{ik}}  \lambda_k II_{ik}, \\
SS'  =& - c SS + \frac{X^2}{2 (X + \sum_k Y_k)}  - \sum_i 2 P_i E_i SS,\\
SI_i' =& - c SI_i + \frac{X Y_i}{X + \sum_k Y_k} - \beta_i SI_i + \sum_k 2 M_{ki} P_k E_k SS\\
&- \sum_k P_k E_k SI_i - \lambda_i SI_i,\\
II_{ij}'& - c II_{ij} +\khalf \frac{Y_i Y_j}{X + \sum_k Y_k} + \khalf \cdot (M_{ij} \beta_i SI_i + M_{ji} \beta_j SI_j)\\
&+ \khalf \cdot (\sum_k (M_{kj} P_k E_k SI_i + M_{ki} P_k E_k SI_j)) -(\lambda_i + \lambda_j) II_{ij}.
\end{aligned}
\end{equation}

\subsection*{Implicit model}

Following \cite{shirreff_transmission_2011}, Model 5 is an implicit instantaneous partnership formation model that uses an adjusted transmission rate, $\beta^\ast$, that is derived from Hollingsworth \etal's approximate basic reproduction number \cite{hollingsworth_hiv1_2008}:

\begin{equation}
\beta^\ast_i = \frac{c \beta_i}{c + \beta_i + \lambda_i}.
\end{equation}
Thus, we get the following model:

\begin{equation}
\begin{aligned}
S' & = \sum_k \lambda_k I_k - \sum_k \beta^\ast_k S I_k.\\
I_i' & = \sum_k M_{ki} \beta^\ast_k S I_k - \lambda_i I_i.
\end{aligned}
\end{equation}

\subsection*{Random-mixing model}

Model 6 is a random-mixing model. It is modeled in a same way as model 5 without the adjusted transmission rate:

\begin{equation}
\begin{aligned}
S' & = \sum_k \lambda_k I_k - \sum_k \beta_k S I_k,\\
I_i' & = \sum_k M_{ki} \beta_k S I_k - \lambda_i I_i.
\end{aligned}
\end{equation}

\subsection*{Heterogenous model}

We extend the ``pairform+epc'' model by allowing for heterogeneity in sexual behaviour. Since pairform+epc model captures four distinct sexual behaviours -- pair formation, pair dissolution, extra-couple mixing, and uncoupled mixing -- we assume that all four parameters that model these behaviours are scaled by the same factor based on the risk group. In other words, an individual in a higher risk is also more likely to form a stable partnership, leave a stable partnership, and engage in a extra-couple/uncoupled mixing. We denote this scaling parameter as $\varphi_i$ where $i$ is the risk group. For simplicity, we assume that the transmission rate per partnership is unaffected by sexual behaviour.

\subsubsection*{Partnership dynamics}

Individuals in a risk group $i$ leave single state at per-person rate $\varphi_i \rho$. Let $XY_{ij,kl}$ be a coupled state where $X$ and $Y$ are the infection status (susceptible or infected) of each partner, $k$ and $l$ are the risk groups $X$ and $Y$ belong to respectively, and $i$ and $j$ are the strains of an infected partner. If a partner is susceptible, strain index is replaced by $\cdot$. For example, $SI_{\cdot j,kl}$ is the number of partners where the susceptible partner is in risk group $k$ and infected partner is in risk group $l$ and has \Lspvl\ of $\alpha_j$. For simplicity, we assume that people undergo random \emph{activity-weighted} mixing \cite{may_transmission_1988}. Then we can write the partnership formation process as follows:

\begin{equation}
\PF(XY_{ij, kl}) = \kkhalf \frac{\varphi_k \rho X_{i,k} \varphi_l \rho Y_{j, l}}{\sum\limits_m \varphi_m \rho (S_{\cdot, m} + \sum\limits_n I_{n,m})}
\end{equation}

For dissolution process, an individual in risk group $i$ leaves its partnership at a rate $\varphi_i c$. If a partnership is formed between two individuals from a different risk group, the rates at which they leave the partnership differ. We resolve this conflict by assuming that a partnership dissolution rate of a couple is equal to the average of that of two partners. Therefore, $XY_{ij, kl}$ dissolve at per-partnership rate $(\varphi_k + \varphi_l)/2 c$, and both $X_{i,k}$ and $Y_{j,l}$ partners return to single state at the same rate.

\subsubsection*{Heterogeneous models: Infection dynamics}

Since we assume that the rate of transmission per partnership stays constant across different risk groups, the within-couple infection process is similar to other models:

\begin{equation}
\begin{aligned}
WT(SI_{\cdot j, kl} &= - \beta_j SI_{\cdot j, kl}\\
WT(II_{ij, kl}) &= \kkhalf \cdot (M_{ji} \beta_j SI_{\cdot j, kl} + M_{ij} \beta_i SI_{\cdot i, lk})
\end{aligned}
\end{equation}

Note that $II_{ij,kl}$ can be formed from two types of partnerships: 1) Infected partner with \Lspvl\ of $\alpha_j$ and risk group of $l$ infects a susceptible partner in risk group $k$, yielding \Lspvl\ of $\alpha_i$ through mutation. 2) Infected partner with \Lspvl\ of $\alpha_i$ and risk group of $k$ infected a susceptible partner in risk group $l$, yielding \Lspvl\ of $\alpha_j$ through mutation. On the other hand, if $i = j$ and $k = l$, $II_{ii,kk}$ can only be formed from $SI_{\cdot i, kk}$ partnership, which is resolved by $\kkhalf$.

The heterogeneous extra-couple and uncoupled contact process is similar to the partnership formation process (activity-weighted random mixing). Relative uncoupled/extra-couple contact rates are scaled by the factor of $\varphi_i$, where $i$ is the risk group. First, we define $Q_{i}$ as the total rate of uncoupled/extra couple contact by individuals in risk group $k$:

\begin{equation}
\begin{aligned}
Q_i = &\varphi_i r_u (S_{\cdot,i} +  \sum_j I_{j,i}) + \varphi_i r_e \bigg( \sum_k 2^{\delta_{ik}} SS_{\cdot, ik} +\\
&\sum_l \sum_j (SI_{\cdot j,il} + SI_{\cdot j, li}) + \sum_j \sum_l \sum_k 2^{\delta_{kl} \delta_{ij}} II_{kl,ij} \bigg)
\end{aligned}
\end{equation}
We now define $P_{k,i}$ as the proportion of the extra-couple and uncoupled contact that arises from an infected individual from risk group $i$ with \Lspvl\ of $\alpha_k$:
\begin{equation}
P_{k,i} = \frac{\varphi_i r_u I_{k,i} + \varphi_i r_e (SI_{k,i} + \sum_j \sum_l 2^{\delta_{kl} \delta_{ij} } II_{kl,ij} )}{\sum_j Q_j}
\end{equation}
Since the relative uncoupled/extra couple contact ratios are scaled by the factor of $\varphi_i$, uncoupled and extra-couple transmission rates are scaled by the same factor as well: $U_{k,i} = \varphi_i r_u \beta_k$ and $E_{k,i} = \varphi_i r_e \beta_k$. Once again, we assume random mixing between individuals. Then, a susceptible individual in risk group $i$ becomes infected through extra-couple and uncoupled contact at a per capita rate of $\sum_j \sum_k P_{k,j} X_{k,i}$. Once infected, individuals are distributed into strain categories through mutation.

\subsubsection*{Heterogeneous model: Disease induced mortality}

Disease induced mortality is not affected by the sexual behaviour of an individual. 

\subsection*{Initial distribution of infected individuals}

We follow Champredon \etal's result to calculate the initial distribution of infected individuals. For model 1 and 2, we have disease equilibrium state of $S^* = \frac{c}{c + \rho}$ and $SS^* = \frac{1-S^*}{2}$. We let $\epsilon = 10^{-4}$, which is the total number of infected individuals in the beginning of simulation and $D$ be the vector such that $D_i$ represent the proportion of individuals with \Lspvl\ of $i$. $Y_i$ is taken from the Normal distribution with mean 3 and is normalized so that $\sum_i D_i = 1$. Then, we have the following initial distribution of each states:

\begin{equation}
\begin{aligned}
S(0) &= (1 - \epsilon) S^*, \\
SS(0) &= (1 - \epsilon)^2 SS^*,\\
SI_i(0) &= 2 \epsilon (1-\epsilon) SS^* D_i,\\
I_i(0) &=  \epsilon S^* D_i,\\
II_{ij}(0) &=  \khalf 2\epsilon^2 SS^* D_i D_j.
\end{aligned}
\end{equation}
Since model 3 and 4 do not have single states, $SS^*=1$ at the disease free equilibrium and the initial distribution becomes:

\begin{equation}
\begin{aligned}
SS(0) &= (1 - \epsilon)^2 SS^*,\\
SI_i(0) &= 2 \epsilon (1-\epsilon) SS^* D_i,\\
II_{ij}(0) &=  \khalf 2\epsilon^2 SS^* D_i D_j.
\end{aligned}
\end{equation}
As model 5 is an implicit model, which does not consider different stages of partnership, we have the following initial distribution:

\begin{equation}
\begin{aligned}
S(0) &= 1 - \epsilon,\\
I_i(0) &=  \epsilon D_i.
\end{aligned}
\end{equation}
Model 6 has the same distribution of initial infected individuals as model 5.

Lastly, for the heterogeneity model, we assume that the risk distribution of the population follows a Gamma distribution and calculate the shape and scale parameters from the mean and squared coefficient of variation. Using the shape and scale parameters, we define the Gamma quantile function $Q(p)$ and $p_j =  \tsub{p}{min} + (\tsub{p}{max} - \tsub{p}{min}) \frac{j-1}{n_r + 1}$, where $n_r$ is number of risk groups and $j = 1, 2, 3, \dots, n_r + 1$. Since $Q(1) = \infty$, we set $\tsub{p}{max} = 0.99$ and $\tsub{p}{min} = 0.01$. Then, we define $\varphi_i = \frac{Q(p_j) + Q(p_{j+1})}{2}$. We define $R_i$ as the proportion of individuals in risk group $i$ at the disease-free equilibrium and assume $R_i$ is equal for all $i$, i.e. $R_i = \frac{1}{n_r}$. In order to start the simulation in a quasi-equilibrium state, we first run the model with the following initial state:

\begin{equation}
\begin{aligned}
S_{\cdot,i}(0) &= (1 - \epsilon) R_i,\\
I_{k,i}(0) &= \epsilon D_k R_i,\\
SS_{\cdot,ij}(0) &= SI_{\cdot k, ij}(0) = II_{kl,ij} (0) = 0.\\
\end{aligned}
\end{equation}
For this particular simulation, we disregard the infection process as well as disease-induced mortality in order to preserve the strain distribution of infected individuals. Furthermore, since the scaling parameter, $\gamma$, does not affect the risk group distribution in the absence of disease transmission, we increase the scaling parameter to 5 ($\gamma = 5$) to speed up the simulation and run the model for 50 years. After the model has reached its quasi-equilibrium state, we take this distribution of susceptible and infected individuals as the initial state of the actual simulation.

\paragraph*{Appendix S2: dynamics of transmission and virulence}

\label{S2_Appendix}

This section presents alternate versions of
the figures from the main text showing time dynamics and summary
statistics in terms of \Lspvl\ and per-year transmission probability
rather than expected progression time to AIDS.

\begin{figure}[!ht]
\includegraphics[width=\textwidth]{../figures/fig_S2_1.pdf}
\caption{{\bf Envelopes of transmission trajectories under all models.}
This figure matches \figurename~\ref{fig:virtraj}, but displays the
envelope of population-mean transmission probabilities rather than \Lspvl\ over time
for each model.
}
\label{fig:transtraj}
\end{figure}

\begin{figure}[!ht]
\includegraphics[width=\textwidth]{../figures/fig_S2_2.pdf}
\caption{{\bf Envelopes of progression trajectories under all models.}
This figure matches \figurename~\ref{fig:virtraj}, but displays the
envelope of population-mean expected time of progression to AIDS (i.e., length of
intermediate HIV phase) rather than \Lspvl\ over time
for each model.
}
\label{fig:durtraj}
\end{figure}

\begin{figure}[!ht]
  \includegraphics[width=\textwidth]{../figures/fig_S2_3.pdf}
\caption{{\bf Univariate distributions of transmission probabilities and progression.}
This figure matches \figurename~\ref{fig:unidist}, but displays the
univariate distributions for the transmission probability and 
progression time at the virulence
peak and at the epidemiological equilibrium,
rather than the distributions of \Lspvl.
}
\label{fig:tranprogsum}
\end{figure}

\clearpage

\nolinenumbers

% Either type in your references using
% \begin{thebibliography}{}
% \bibitem{}
% Text
% \end{thebibliography}
%
% or
%
% Compile your BiBTeX database using our plos2015.bst
% style file and paste the contents of your .bbl file
% here.
% 

\bibliography{virulence}
\end{document}

