\documentclass[10pt]{letter}

\usepackage{graphicx}
\usepackage{myltr2}
\usepackage{times}
%\usepackage[sort&compress]{natbib}

\bibliographystyle{abbrv}

\makeatletter
% http://tex.stackexchange.com/questions/18033/using-bibtex-with-letter-class
\newenvironment{thebibliography}[1]
     {\list{\@biblabel{\@arabic\c@enumiv}}%
           {\settowidth\labelwidth{\@biblabel{#1}}%
            \leftmargin\labelwidth
            \advance\leftmargin\labelsep
            \usecounter{enumiv}%
            \let\p@enumiv\@empty
            \renewcommand\theenumiv{\@arabic\c@enumiv}}%
      \sloppy
      \clubpenalty4000
      \@clubpenalty \clubpenalty
      \widowpenalty4000%
      \sfcode`\.\@m}
     {\def\@noitemerr
       {\@latex@warning{Empty `thebibliography' environment}}%
      \endlist}
\newcommand\newblock{\hskip .11em\@plus.33em\@minus.07em}
\makeatother

\begin{document}

\date{\today}

\signature{\includegraphics[height=0.35in]{bbsig3.png}\\Benjamin Bolker}

\begin{letter}{
} 

\opening{To the editor,}

We (Daniel Park and Ben Bolker) are submitting our manuscript, ``Effects of Epidemiological Structure on the Transient Evolution of HIV Virulence'', for consideration for publication in \emph{PLoS Computational Biology}. Using HIV as a model system, our study explores the transient, eco-evolutionary changes in pathogen characteristics over the course of an epidemic. Building on modeling frameworks and estimated parameters developed by Shirreff \emph{et al.} \cite{shirreff_transmission_2011} and Champredon \emph{et al.} \cite{champredon_hiv_2013}, we ask how predicted changes in virulence depend on way that contact patterns (e.g. the dynamics of sexual partnership change and contact outside of stable partnerships) are modeled.

This study is important because researchers are actively developing --- and publishing in high-impact journals --- studies that predict the evolutionary consequences of public-health interventions \cite{payne_impact_2014,roberts2015impact,herbeck2016evolution}. Our work suggests that basic epidemiological structures that are neglected in these models can lead to two- to fourfold differences in predictions of outcomes such as transmission probability or expected time to progress to AIDS.

%suggested reviewers: Christophe Fraser, Shirreff, Samuel Alizon, Sebastian Lion, Troy Day

\closing{Sincerely,}

\bibliography{virulence}

\end{letter}
\end{document}
